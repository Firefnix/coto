\section{Abstract states}

We now have intervals, \textit{abstract elements} of $\mathbb{C}$ represented in $\mathcal{A}_0$. Our abstract elements for a $n$-qubit quantum state would be in $\mathcal{A}_n = {\mathcal{A}_0}^{2^n}$ for all $n \in \mathbb{N}$. Defining a sum in $\mathcal{A}_n$, and an external product $\alpha * A$ for
$\alpha \in \mathcal{A}_0$ and $A \in \mathcal{A}_n$, comes easily. We also define the inclusion relation in $\mathcal{A}_n$ (which is an order relation) $\subset$ using the cartesian product of sets:
$$\forall A = (a_0, ..., a_{2^n-1})^{T},  B = (b_0, ..., b_{2^n-1})^{T} \in \mathcal{A}_n, A \subset B \iff a_0 \times ... \times a_{2^n-1} \subset b_0 \times ... \times b_{2^n-1}$$

Intuitively, if $A$ and $B$ are in $\mathcal{A}_n$ and $A \subset B$, the abstract state $A$ is more \textit{precise} than $B$. In order to work with this notion, and later on to make it work with diagrams and approximations, we define the \textit{imprecision} $\mathcal{I}$ of an abstract state:
$$\function{\mathcal{I}}
{\bigcup_{n \in \mathbb{N}} \mathcal{A}_n}{\mathbb{R}^+}
{A \in \mathcal{A}_n}{\displaystyle\prod_{i = 0}^{2^n-1} s(A_i)}$$

\noindent using the surface function $s$ on complex intervals defined by
$$s(\alpha) = (\max\{\RE(z) ; z \in \alpha\} - \min\{\RE(z) ; z \in \alpha\}) \times (\max\{\IM(z) ; z \in \alpha\} - \min\{\IM(z) ; z \in \alpha\})$$

\section{Decision diagrams}

We inductively define abstract additive quantum decision diagrams, starting from zero-depth decision. The only zero-depth diagram is $\boxed{1}$. Let for every set $E$ be the set of finite subsets of $E$: $\mathscr{P}_f(E) = \{ A \subset E / |A| < \infty \}$. If the set $\mathcal{D}_n$ of diagrams of depth $n$ is defined, $n+1$-depth diagrams can have a finite number of left children in $\mathcal{D}_n$ and a finite number of right children in $\mathcal{D}_n$, each being associated with an abstract amplitude in $\mathcal{A}_0$.

\begin{center}
    % https://tikzcd.yichuanshen.de/#N4Igdg9gJgpgziAXAbVABwnAlgFyxMJZAZgBoAGAXVJADcBDAGwFcYkQAREAX1PU1z5CKcqQCM1Ok1bsoAfXI8+IDNjwEiY8ZIYs2iEADpjS-mqFEATNpq6ZB+cDABaMd1MqB64cgAsNqT12AHM5MQ9VQQ0UAFYAu30jE14zKJ8ANnjpRNDgAFtXd25JGChg+CJQADMAJwg8pFEQHAgkMkD7EAAdLqY0AAt6OSdCkBpGegAjGEYABS8LA0YYKpwxkEYsMESoejh+0o9a+saaFqR-DsSe6ZwhxXGpmfnzaJAarGD+tZSQY4bEE1zohMld2DcYHdhgU3OsJtM5gs3stVkc6gCga1EFowQYen1Bgo4U9Ea9hO9Pt84VsdnsDlAeJRuEA
    \begin{tikzcd}
        &     &         & D \arrow[ld, "\alpha_{n-1}", dashed] \arrow[rd, "\beta_0"'] \arrow[rrrd, "\beta_{m-1}"] \arrow[llld, "\alpha_0"', dashed] &     &     &         \\
    d_0 & ... & d_{n-1} &                                                                                                                           & g_1 & ... & g_{m-1}
    \end{tikzcd}
\end{center}

Defining $\mathcal{D}_{n+1} = \mathscr{P}_f(\mathcal{A}_0 \times \mathcal{D}_n) \times \mathscr{P}_f(\mathcal{A}_0 \times \mathcal{D}_n)$ thus comes naturally. Eventually, let $\mathcal{D} = \bigcup_{n \in \mathbb{N}} \mathcal{D}_n$ the set of all AAQDDs.

\section{Sub-diagrams}

Sub-diagrams (or nodes) of a diagram $D \in \mathcal{D}_n$ for all $n \in \mathbb{N}$, are defined inductively:
$$\mathcal{N}(\boxed{1}) = \boxed{1}$$
$$\forall D, G \in \mathscr{P}_f(\mathcal{A}_0 \times \mathcal{D}_n), \mathcal{N}(D, G) = \{d ; (\alpha, d) \in D \cup G\} \cup
\bigcup_{(\alpha, d) \in D \cup G} \mathcal{N}(d)$$

We also define $\mathcal{N}_i(D) = \mathcal{N}(D) \cap \mathcal{D}_i$ for all $i \in \mathbb{N}$.

\section{Diagram evaluation}

Now that we defined our decision diagrams, we can evaluate them to get abstract elements. We inductively define our evaluation function for $n$ quibits $\mathcal{E}_n : \mathcal{D}_n \rightarrow \mathcal{A}_n$:

$$\mathcal{E}_0(\boxed{1}) = \{1\}$$

$$\forall D, G \in \mathscr{P}_f(\mathcal{A}_0 \times \mathcal{D}_n), \mathcal{E}_{n+1}(D, G) =
\begin{pmatrix}
    \displaystyle\sum_{(\alpha, g) \in G} \alpha * \mathcal{E}_n(g) \\
    \displaystyle\sum_{(\beta, d) \in D} \beta * \mathcal{E}_n(d) \\
\end{pmatrix}
\quad\text{with}
$$

Since there is no risk of ambiguity, defining $\mathcal{E} : \bigcup \mathcal{D}_n \rightarrow \bigcup \mathcal{A}_n$ is not problematic. With this last function, we can now evaluate all our AAQDDs and define a partial order $\le$ on $\mathcal{D}$, with:
$$\forall A, B \in \mathcal{D}, A \le B \iff \mathcal{E}(A) \subset \mathcal{E}(B)$$

Additionally, we extend the definition of the imprecision function $\mathcal{I}$ to diagrams: $\forall D \in \mathcal{D}, \mathcal{I}(D) = \mathcal{I}(\mathcal{E}(D))$. Note that $\forall A, B \in \mathcal{D}_n, A \le B$ implies that $\mathcal{I}(A) \le \mathcal{I}(B)$ but that the reciprocal is not generally true.
