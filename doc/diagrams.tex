\section{Abstract states}

We now have intervals, \textit{abstract elements} of $\mathbb{C}$ represented in $\mathcal{A}_0$. Our abstract elements for a $n$-qubit quantum state would be in $\mathcal{A}_n = {\mathcal{A}_0}^{2^n}$ for all $n \in \mathbb{N}$. Defining a sum in $\mathcal{A}_n$, and an external product $\alpha * A$ for
$\alpha \in \mathcal{A}_0$ and $A \in \mathcal{A}_n$, comes easily. We also define the inclusion relation in $\mathcal{A}_n$ (which is an order relation) $\subset$ using the cartesian product of sets:
$$\forall A = (a_0, ..., a_{2^n-1})^{T},  B = (b_0, ..., b_{2^n-1})^{T} \in \mathcal{A}_n, A \subset B \iff a_0 \times ... \times a_{2^n-1} \subset b_0 \times ... \times b_{2^n-1}$$

Intuitively, if $A$ and $B$ are in $\mathcal{A}_n$ and $A \subset B$, the abstract state $A$ is more \textit{precise} than $B$.

\section{Decision diagrams}

We inductively define abstract additive quantum decision diagrams (AAQDDs), starting from zero-depth decision. The only zero-depth diagram is $\boxed{1}$. Let for every set $E$ be the set of finite subsets of $E$: $\mathscr{P}_f(E) = \{ A \subset E / |A| < \infty \}$. If the set $\mathcal{D}_n$ of diagrams of depth $n$ is defined, $n+1$-depth diagrams can have a finite number of left children in $\mathcal{D}_n$ and a finite number of right children in $\mathcal{D}_n$, each being associated with an abstract amplitude in $\mathcal{A}_0$.

\begin{center}
    % https://tikzcd.yichuanshen.de/#N4Igdg9gJgpgziAXAbVABwnAlgFyxMJZAZgBoAGAXVJADcBDAGwFcYkQAREAX1PU1z5CKcqQCM1Ok1bsoAfXI8+IDNjwEiY8ZIYs2iEADpjS-mqFEATNpq6ZB+cDABaMd1MqB64cgAsNqT12AHM5MQ9VQQ0UAFYAu30jE14zKJ8ANnjpRNDgAFtXd25JGChg+CJQADMAJwg8pFEQHAgkMkD7EAAdLqY0AAt6OSdCkBpGegAjGEYABS8LA0YYKpwxkEYsMESoejh+0o9a+saaFqR-DsSe6ZwhxXGpmfnzaJAarGD+tZSQY4bEE1zohMld2DcYHdhgU3OsJtM5gs3stVkc6gCga1EFowQYen1Bgo4U9Ea9hO9Pt84VsdnsDlAeJRuEA
    \begin{tikzcd}
        &     &         & D \arrow[ld, "\alpha_{n-1}", dashed] \arrow[rd, "\beta_0"'] \arrow[rrrd, "\beta_{m-1}"] \arrow[llld, "\alpha_0"', dashed] &     &     &         \\
    d_0 & ... & d_{n-1} &                                                                                                                           & g_1 & ... & g_{m-1}
    \end{tikzcd}
\end{center}

Defining $\mathcal{D}_{n+1} = \mathscr{P}_f(\mathcal{A}_0 \times \mathcal{D}_n) \times \mathscr{P}_f(\mathcal{A}_0 \times \mathcal{D}_n)$ thus comes naturally. Eventually, let $\mathcal{D} = \bigcup_{n \in \mathbb{N}} \mathcal{D}_n$ the set of all AAQDDs.

\section{Sub-diagrams}

Sub-diagrams (or nodes) of a diagram $D \in \mathcal{D}_n$ for all $n \in \mathbb{N}$, are defined inductively:
$$\mathcal{N}(\boxed{1}) = \boxed{1}$$
$$\forall D, G \in \mathscr{P}_f(\mathcal{A}_0 \times \mathcal{D}_n), \mathcal{N}(D, G) = \{d ; (\alpha, d) \in D \cup G\} \cup
\bigcup_{(\alpha, d) \in D \cup G} \mathcal{N}(d)$$

We also define $\mathcal{N}_i(D) = \mathcal{N}(D) \cap \mathcal{D}_i$ for all $i \in \mathbb{N}$ the nodes "at height $i$".

\section{Diagram evaluation}

Now that we defined our decision diagrams, we can evaluate them to get abstract elements. We inductively define our evaluation function for $n$ quibits $\mathcal{E}_n : \mathcal{D}_n \rightarrow \mathcal{A}_n$:

$$\mathcal{E}_0(\boxed{1}) = \{1\}$$

$$\forall D, G \in \mathscr{P}_f(\mathcal{A}_0 \times \mathcal{D}_n), \mathcal{E}_{n+1}(D, G) =
\begin{pmatrix}
    \displaystyle\sum_{(\alpha, g) \in G} \alpha * \mathcal{E}_n(g) \\
    \displaystyle\sum_{(\beta, d) \in D} \beta * \mathcal{E}_n(d) \\
\end{pmatrix}
\quad\text{with}
$$

Since there is no risk of ambiguity, defining $\mathcal{E} : \bigcup \mathcal{D}_n \rightarrow \bigcup \mathcal{A}_n$ is not problematic. With this last function, we can now evaluate all our AAQDDs and define a partial order $\le$ on $\mathcal{D}$, with:
$$\forall A, B \in \mathcal{D}, A \le B \iff \mathcal{E}(A) \subset \mathcal{E}(B)$$

Additionally, we extend the definition of the imprecision function $\mathcal{I}$ to diagrams: $\forall D \in \mathcal{D}, \mathcal{I}(D) = \mathcal{I}(\mathcal{E}(D))$. Note that $\forall A, B \in \mathcal{D}_n, A \le B$ implies that $\mathcal{I}(A) \le \mathcal{I}(B)$ but that the reciprocal is not generally true.

\section{Error estimation}

Let $A, C \in \mathcal{D}_n$. We would like to define a scale $\delta : \mathcal{D}_n \rightarrow X$ with $(X, \prec)$ a partially ordered set, that we would be able to compute in polynomial time and space (i.e. in $O(n^p)$ for a certain $p \in \mathbb{N}$), such that if $A \le C$, $\delta(A) \prec \delta(C)$. Let $X = \mathbb{R}^+ \times \mathcal{A}_0$ and $\delta(A) = (|A|, [A])$ be inductively defined by:
$$\delta(\boxed{1}) = (1, [1, 1])$$
$$\forall a, b \in \mathcal{A}_0, \delta(\{(a, \boxed{1})\}, \{(b, \boxed{1})\}) = $$
$$\forall G, D \in \mathscr{P}_f(\mathcal{A}_0 \times \mathcal{D}_n), |(G, D)|
= \max\left( \left|\sum_{(l, L) \in G} l |L| \right|, \left|\sum_{(r, R) \in D} r |R| \right|\right)$$
$$\forall G, D \in \mathscr{P}_f(\mathcal{A}_0 \times \mathcal{D}_n), [(G, D)]
= \left(\sum_{(l, L) \in G} l |L| [L] \right) \bigsqcup \left(\sum_{(r, R) \in D} r |R| [R]\right)$$

\noindent\underline{\textbf{Proof:}} Let for $n \ge 0$ the property
$$(\Delta_n)\quad : \quad \forall A, B \in \mathcal{D}_n, A \le B \Rightarrow \delta(A) \prec \delta(B)$$

\noindent\boxed{n = 0} The only zero-height diagram is $\boxed{1}$, the result is trivial.

\noindent\boxed{n = 1} Let $A, B \in \mathcal{D}_1$.

\begin{center}
% https://tikzcd.yichuanshen.de/#N4Igdg9gJgpgziAXAbVABwnAlgFyxMJZABgBpiBdUkANwEMAbAVxiRAEEQBfU9TXfIRQAmclVqMWbAELdeIDNjwEiZAIzj6zVohAAdPQCMIADxhRgarnL5LBRURupapug8bMWr3ceYDm8ESgAGYAThAAtkhkIDgQSKIS2mx0APoMAH4ZINQMdIYwDAAK-MpCIKFYfgAWODkgDFhgOiBQdHDV5vUFYFBIALQALACcPCHhUYgxcQnOki1ZaaH1eQXFpfa6DDDBddQ9fYgjYyBhkUhq1DOIAMxzybqG6Vkr+YUldiq6lTV7DU0tNodLr7GC9AbHeRnSaXWLxW73VwgLJPZa5N7rT7lba7bpgw7HChcIA
\begin{tikzcd}
A \arrow[d, "a_l~~"', dashed, bend right=49] \arrow[d, "~~a_r", bend left=49] &  & B \arrow[d, "b_l~~"', dashed, bend right=49] \arrow[d, "~~b_r", bend left=49] \\
\boxed{1}                                                                     &  & \boxed{1}
\end{tikzcd}
\end{center}

Suppose that $A \le B$, then $a_l \le b_l$ and $a_r \le b_r$, hence $a_l [\boxed{1}] \sqcup a_r [\boxed{1}] = a_l + a_r \le b_l + b_r$, and then $\delta(A) \prec \delta(B)$, so $(\Delta_1)$ is true.

\noindent\boxed{n > 1} Let $n > 1$ such that $(\Delta_{n-1})$ is true. Let $A, B \in \mathcal{D}_n$ such that $A \le B$.

% https://tikzcd.yichuanshen.de/#N4Igdg9gJgpgziAXAbVABwnAlgFyxMJZAJgBoAGAXVJADcBDAGwFcYkQBBEAX1PU1z5CKAKwVqdJq3YAhHnxAZseAkXKkAjBIYs2iEABkA+hvn9lQohs3apew0cK9zg1SjJaaO6foBKJs0UBFWFkAGYbLzt2fwBbQKVXUIAWSMlddmNTZyCLN2QxT3SfBycFRJCiADY073t-bPLgyxQAdlrovyN47gkYKABzeCJQADMAJwh4xHUQHAgkCOK9MGZGRhpGegAjGEYABWa3EEYYUZwQTawweyh6OAALfsCJqaRZ+aRU5aRV9c2dntDnlhCczhccq9ph8FogxD9EH8NidAQcjqDTucXpNoTRPogyAikQDdmiQexMRcrjd2HdHs9ITikNY5rCakS1sitqTgUkKeDLidrrd7k8oNi3ogWfj2hz-iieej+VjqSL6eLGZLpbCABxRDKIzkkoFK-SUiXTbVIACc+p8xIVJvJZoFmumhPxSzqvyNjrJfJdKpAT3o4v0kBpqtpEBwOAZCihXzxsPh3sN8u5ToDYKDIbD4AIbCj+igMbjGoTTMQ7JldpWvsz-sqgapwZgofYEaLQppJbL8bGVb1rJtdZ9GdRvObOdbec7hcFjGF0djDMo3CAA
\begin{tikzcd}
                               &     & A \arrow[ld, dashed] \arrow[d] \arrow[rd] \arrow[lld, dashed] &     &                                & B \arrow[ld, dashed] \arrow[d, dashed] \arrow[rd] \arrow[rrd] &                                &     \\
L_1 \arrow[r, no head, dotted] & L_n & R_1 \arrow[r, no head, dotted]                                & R_m & L_1 \arrow[r, no head, dotted] & L_n                                                           & R_1 \arrow[r, no head, dotted] & R_m
\end{tikzcd}

Let $a_j = \text{ampl}(j, A)$ and $b_j = \text{ampl}(j, B)$. We know that $A \le B$ so:
$$\sum_j a_j \mathcal{E}(j) \le \sum_j b_j \mathcal{E}(j)$$

We would like to prove that:
$$|A \le |B|, ~\text{i.e.}~ \max()$$

\section{Non-additive diagrams}

We can also define non-additive diagrams, which are diagrams that have at most one left child and one right child. We define $\mathcal{D}^* = \bigcup_{n \in \mathbb{N}} \mathcal{D}^*_n$ the set of all non-additive diagrams, with $\mathcal{D}^*_n = \mathscr{S}(\mathcal{A}_0 \times \mathcal{D}^*_{n-1}) \times \mathscr{S}(\mathcal{A}_0 \times \mathcal{D}^*_{n-1})$ and $\mathscr{S}(E) = \{ \{x\} ; x \in E\} \cup \{\emptyset\}$. As we will see in the next chapter, non-additive diagrams have convenient properies that are not generally verified in additive diagrams, enabling us to provide a reduction algorithm.

\begin{wrapfigure}{r}{0pt}
% https://tikzcd.yichuanshen.de/#N4Igdg9gJgpgziAXAbVABwnAlgFyxMJZABgBpiBdUkANwEMAbAVxiRAEEB9QgX1PUy58hFGQCMVWoxZsAdPJB8B2PASJkATJPrNWiDpzGL+IDCuHrSAZm3S9B4gAIAvI4A6bgEYQAHjCjAYjyKkv4A5vBEoABmAE4QALZIZCA4EEhi1DoyiGBMDAzUDHSeMAwACoKqIiAMMNE4IEVYYPZQdHAAFv5NIKVgUEgAtFbESiBxicnUaRlZdkh5BUUlZZXmavp1Db39g4ij45NJiJmp6Yga87qL+YW1qxVVFlv1jc2tbO1dPdR7w4cTMc5uckFcpDdcncVqUnhsatt3n0YAMkICYvETuDZgdrjklvdirD1kJNrU3r0GC02h1uoM-ij9iMxkDMWCZhcrHi9ASYWtnmTEbtGWixhQeEA
\begin{tikzpicture}
\begin{tikzcd}
    A_n \arrow[d, dashed, bend right] \arrow[d, bend left] \\
    ... \arrow[d, dashed, bend right] \arrow[d, bend left] \\
    A_1 \arrow[d, dashed, bend right] \arrow[d, bend left] \\
    A_0 = \boxed{1}                                       
\end{tikzcd}
\end{tikzpicture}
\end{wrapfigure}

Even more specifically, we define abstract chains as diagrams that have at most one left child and one right child, and if having two children, those two children being equal. We define $\mathcal{D}^{\dagger} = \bigcup_{n \in \mathbb{N}} \mathcal{D}^{\dagger}_n$ the set of all abstract chains, with $\mathcal{D}^{\dagger}_n$ the set of abstract chains of height $n$.
