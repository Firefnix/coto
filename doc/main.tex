\documentclass{article}

\usepackage[english]{babel}
\usepackage[utf8]{inputenc}
\usepackage{geometry}
\usepackage{amsmath}
\usepackage{amssymb}
\usepackage{mathrsfs}

\usepackage{tikz}
\usetikzlibrary{positioning}
\usepackage{tikz-cd}


\newcommand{\fonction}[5]{\begin{array}{l|rcl}
#1: & #2 & \longrightarrow & #3 \\
    & #4 & \longmapsto & #5 \end{array}}

\title{AAQDD -- Abstract additive quantum decision diagrams}
\author{M. Leroy, R. Vilmart}

\begin{document}

\maketitle

\section{Formal definition}

\subsection{Abstract states}

The purpose of quantum decision diagrams is to provide a more efficient way to store and manipulate quantum states of a finite number of qubits. A $n$-qubit state is indeed traditionally represented as an element of $\mathbb{C}^{2^n}$ (with norm 1), which takes exponential space as $n$ grows. Abstract states will be in this part defined similarly, but with complex intervals instead of complex numbers.

The standard definition of real intervals is:

$$\forall a, b \in \mathbb{R}, [a, b]
= \{x \in \mathbb{R} / \min(a, b) \le x \le \max(a, b)\}$$

Those can be generalised to complex intervals, commonly using the cartesian notation.

$$\forall x, y \in \mathbb{C}, [x, y]
= \{a + ib ; a \in [\Re(x), \Re(y)], b \in [\Im(x), \Im(y)]\}$$

Now let $\mathcal{A}_0 = \{[x, y] ; x, y \in \mathbb{C}\}$, we can now define basic operations: sum, product, and join. We note that these definition do not depend on the type of complex interval chosen.
$$\forall \alpha, \beta \in \mathcal{A}_0, \alpha + \beta = \{a + b ; a \in \alpha, b \in \beta\}$$
$$\forall \alpha, \beta \in \mathcal{A}_0, \alpha * \beta = \{a * b ; a \in \alpha, b \in \beta\}$$
$$\forall \alpha, \beta \in \mathcal{A}_0, \alpha \sqcup \beta = \bigcap_{\alpha \subset \gamma~\text{and}~\beta \subset \gamma} \gamma$$

We now have intervals, \textit{abstract elements} of $\mathbb{C}$ represented in $\mathcal{A}_0$. Our abstract elements for a $n$-qubit quantum state would be in $\mathcal{A}_n = {\mathcal{A}_0}^{2^n}$ for all $n \in \mathbb{N}$. Defining a sum in $\mathcal{A}_n$, and an external product $\alpha * A$ for
$\alpha \in \mathcal{A}_0$ and $A \in \mathcal{A}_n$, comes easily.

\subsection{Decision diagrams}

We inductively define abstract additive quantum decision diagrams, starting from zero-depth decision. The only zero-depth diagram is $\boxed{1}$. Let for every set $E$ be the set of finite subsets of $E$: $\mathscr{P}_f(E) = \{ A \subset E / |A| < \infty \}$. If the set $\mathcal{D}_n$ of diagrams of depth $n$ is defined, $n+1$-depth diagrams can have a finite number of left children in $\mathcal{D}_n$ and a finite number of right children in $\mathcal{D}_n$, each being associated with an abstract amplitude in $\mathcal{A}_0$.

\begin{center}
    % https://tikzcd.yichuanshen.de/#N4Igdg9gJgpgziAXAbVABwnAlgFyxMJZAZgBoAGAXVJADcBDAGwFcYkQAREAX1PU1z5CKAEykAjNTpNW7KAH1gYALTjuPPiAzY8BIgBYJUhizaIQAc3niN-HUKIA2IzROzzV4AFtV63ncE9FHIXaVM5eXJbLQFdYWRxULczEAA6dOjtQPiAViSZFPTUnikYKAt4IlAAMwAnCC8kEJAcCCREsPcQAB1upjQAC3pFFTUQGkZ6ACMYRgAFWIdzRhhqnGi6hqaaVqQxTpTemZxhqInp2YX7IJBarAsB9f8QTcbEZt3EMgP2I5gTxQ+MbPV7bFptRCGH7mXr9IaRcYgSYzeaLG53B5PSjcIA
    \begin{tikzcd}
        &     &         & D \arrow[ld, "\alpha_{n-1}"] \arrow[rd, "\beta_0"'] \arrow[rrrd, "\beta_{m-1}"] \arrow[llld, "\alpha_0"'] &     &     &         \\
    d_0 & ... & d_{n-1} &                                                                                                           & g_1 & ... & g_{m-1}
    \end{tikzcd}
\end{center}

Defining $\mathcal{D}_{n+1} = \mathscr{P}_f(\mathcal{A}_0 \times \mathcal{D}_n) \times \mathscr{P}_f(\mathcal{A}_0 \times \mathcal{D}_n)$ thus comes naturally.

\subsection{Diagram evaluation}

Now that we defined our decision diagrams, we can evaluate them to get abstract elements. We inductively define our evaluation function for $n$ quibits $\mathcal{E}_n : \mathcal{D}_n \rightarrow \mathcal{A}_n$:

$$\mathcal{E}_0(\boxed{1}) = \{1\}$$

$$\forall D, G \in \mathscr{P}_f(\mathcal{A}_0 \times \mathcal{D}_n), \mathcal{E}_{n+1}(D, G) =
\begin{pmatrix}
    \displaystyle\sum_{(\alpha, g) \in G} \alpha * \mathcal{E}_n(g) \\
    \displaystyle\sum_{(\beta, d) \in D} \beta * \mathcal{E}_n(d) \\
\end{pmatrix}
\quad\text{with}
$$

Since there is no risk of ambiguity, defining $\mathcal{E} : \bigcup \mathcal{D}_n \rightarrow \bigcup \mathcal{A}_n$ is not problematic. With this last function, we can now evaluate all our AAQDDs.



\end{document}
