We note that multiple QDDs can be evaluated to the same abstract state. This part will aim to provide an algorithm to decrease the "size" of diagrams while not breaking their evaluation by $\mathcal{E}$. The size of a AAQDD is the number of intervals (counting them with their multiplicity). We would want, from a diagram $D$, to get a diagram $D'$ such that $\text{size}(D) > \text{size}(D')$ and $D \le D'$ (the reduction is smaller in size and might be less precise that the original diagram). More generally, function $g : \mathcal{D}_n \rightarrow \mathcal{D}_n$ is an \textit{global approximation} if
$$\forall D \in \mathcal{D}_n, D \le g(D)$$

Similarly, a function $f : \mathcal{D}_n \times \mathcal{D}_n \rightarrow \mathcal{D}_n$ is a \textit{merge approximation at height $n$} if
$$\begin{cases}
    \forall A \not= B \in \mathcal{D}_n, A \le f(A, B)~\text{and}~B \le f(A, B) \\
    \forall A \in \mathcal{D}_n, f(A, A) = A
\end{cases}
$$

\section{Abstract quantum decision diagrams}

Non-additive abstract quantum decision diagrams, because they only have at most one left child and one right child, are easier to manipulate than AAQDDs. In this section we show, given an AQDD $A$ of height $n$ and $n$ integers $m_1, ..., m_n$ an algorithm to get an AQDD $A'$ of height $n$ such that $D \le A'$ and that $\forall i \in \{0, ..., n\}, |\mathscr{N}_i(A')| \le |\mathscr{N}_i(A)|$. This implies that $\text{size}(A) > \text{size}(A')$.


This algorithm relies on the following transofrmation, that allows us to merge two nodes.
\begin{center}
% https://tikzcd.yichuanshen.de/#N4Igdg9gJgpgziAXAbVABwnAlgFyxMJZAJgBoAGAXVJADcBDAGwFcYkQBBEAX1PU1z5CKAGwVqdJq3YcA5Dz4gM2PASIBGUuokMWbRCAAyC-iqFEAzFp1T9IAEomlA1cOQAWazV3SDT5YJqKACsXpJ67P4u5qJhPnYAGlFmQcjkpMQ2EQYAqsmBbpqZ3rbsAGr5rpYZWb4gebymBUSexeF1FY3OKW6hbfHsDYoBVSgA7DUl2SCdEjBQAObwRKAAZgBOEAC2SOkgOBBIZO12HAAEADoX61gLABY49OubAO5nxjSM9ABGMIwACtEgiAbvccCBPlgwHYoPQ4Hd5k4NttdjQDkgrCcZJdrrcHk9XmdHJ8fn9AT12IwYKtwV1kTtEJp9odEGIsQZzldQfjnhA3h8QF9fgCgcIQXjwZDoexYfDETRfmAoEgALTuACcdM2DKZ6NZUzqHC5EoJfKJEMFpJFFIMVJpFsVysQGq1KMQxz1E3ZRhx3MevLeeRJwvJzQMfotjChMLhCOVroZHpZAA4DXZDL6TQGzhVg2TRZTqbTFPSMWiWeq0+x7JmwabA5GraHRuKwZHozLY4iE2XmUh1HsBgYa8a69nc5aQwXbUWkdqkJ4+4hQuywMxGIw89aw4LZ1KYRBmN8qXO3Wy9eomUOQAlazzCUHJ-mba2Hu3pQZZXHTwzzyz1Mc163qO95mhOQrPjudq0pQ3BAA
\begin{tikzcd}
    &                                                                       & A \arrow[ld, "A \rightarrow L"', dashed] \arrow[rd, "A \rightarrow R"] &                                                                       &                          &    & A' \arrow[d, "A \rightarrow L"', dashed, bend right=49] \arrow[d, "A\rightarrow R", bend left=49] &   \\
    & L \arrow[ld, "L \rightarrow U"', dashed] \arrow[d, "L \rightarrow V"] &                                                                        & R \arrow[d, "R \rightarrow U"', dashed] \arrow[rd, "R \rightarrow V"] & {} \arrow[r, Rightarrow] & {} & X \arrow[ld, "X \rightarrow U"', dashed] \arrow[rd, "X \rightarrow V"]                            &   \\
  U & V                                                                     &                                                                        & U                                                                     & V                        & U  &                                                                                                   & V
  \end{tikzcd}
\end{center}
\vspace{1em}
\noindent with $X \rightarrow U = (L\rightarrow U) \sqcup (R\rightarrow U)$ and $X \rightarrow V = (L\rightarrow V) \sqcup (R\rightarrow V)$. With $X = f_m(L, R)$, such a defined function $f_m$ is a merge approximation, which makes $A' = f(A, L, R)$ an approximation of $A$ using the merging theorem.

\begin{algorithm}[H]
  \caption{Chaining of non-additive AQDDs}
  \begin{algorithmic}
  \Function{chain}{$A$}
  \State $L \gets \text{left\_child}(A)$
  \State $R \gets \text{right\_child}(A)$
  \If{$\text{children}(L) = \text{children}(R)$}
    \State $A \gets f(A, L, R)$
    \State $\text{child}(A) \gets \textsc{chain}(\text{child}(A))$
  \Else \Comment{We can do better by reducing one child and retrying to merge}
    \State $L \gets \textsc{chain}(L)$
    \State $R \gets \textsc{chain}(R)$
    \State $A \gets \mathsc{zip}(f(A, L, R))$
  \EndIf
  \State \Return $A$
  \EndFunction

  \Function{zip}{$n ; k ; L_k, ..., L_1 ; R_k, ..., R_1 ; X$}
  \If{$k = 0$}
    \State \Return $X$
  \Else
  \State \Return $\mathsc{zip}(n ; k ; L_k, ..., L_2 ; R_k, ..., R_2 ; f_m(L_1, R_1)) + (X)$
  \EndIf
  \EndFunction
\end{algorithmic}
\end{algorithm}

\begin{theorem}[Chaining of non-additive AQDDs]
  $$\forall n \in \mathbb{N}, \begin{cases}
    \mathsc{chain}(A) \text{ is a chain} \\
    \mathsc{chain}(A) \le A \\
  \end{cases}$$
\end{theorem}
\begin{proof}
  The proof is done by induction on the height of the AQDD $A$. The base case is trivial, as the AQDD is a leaf and the algorithm returns the same AQDD. Let us assume that the theorem holds for AQDDs of height $n-1$. We will show that it holds for AQDDs of height $n$.

  \underline{Case 1:} The children of $A$ are the same. In this case $A'$ is
% https://tikzcd.yichuanshen.de/#N4Igdg9gJgpgziAXAbVABwnAlgFyxMJZABgBpiBdUkANwEMAbAVxiRAEEByEAX1PUy58hFGQCMVWoxZsAOrJwwAHjjgBjYGoAWdLGB4AKAGYB9ALYGAMqQAEAJQCUD3pJhQA5vCKgjAJwhmSGQgOBBIYtT0zKyIHDbyvljuWjh0vv4A7jaWINQMdABGMAwACoJ4BGyJyTi5IAx6MSBQdHBabnVFYFBIALQALACcfD7+gYjBoeGR0k3s8bLVKWmZ9nX5RaXlwmwMMEa11F09iEM8FDxAA
\begin{tikzcd}
  A' \arrow[d, "A \rightarrow L"', dashed, bend right=49] \arrow[d, "A \rightarrow R", bend left=49] \\
  {\textsc{chain}(f_m(L, R))}             
\end{tikzcd}

Since $f_m(L, R) \in \mathcal{D}_{n-1}$, by induction $\textsc{chain}(f_m(L, R))$ is a chain of AQDDs of height $n-1$. Additionally
$$\begin{cases}
  \mathsc{chain}(f_m(L, R)) \le f_m(L, R) \le L \\
  \mathsc{chain}(f_m(L, R)) \le f_m(L, R) \le R
\end{cases}
$$
\noindent so $(A \rightarrow L) \mathsc{chain}(f_m(L, R)) \le A \rightarrow L) L$ and $(A \rightarrow R) \mathsc{chain}(f_m(L, R)) \le A \rightarrow R) R$.

\underline{Case 2:} The children of $A$ are different. In this case, we have that $A'$ is

\begin{center}
% https://tikzcd.yichuanshen.de/#N4Igdg9gJgpgziAXAbVABwnAlgFyxMJZABgBoAmAXVJADcBDAGwFcYkQAZAfWDAFoAjAF8QQ0uky58hFOQrU6TVuwBKPfsNHiQGbHgJEyAZgUMWbRCAB0NrRL3SickzTPLLNq3Z2T9M5AKkACymShacXMQABAC8UQA68QBGEAAeMFDAwrFRasTeulIGKIHEoebsAIIA5AW+jihkAuXuEWA5iQC29DgAFnAAxsADvfRYYEIAFBwAlHUOxchyza5hqlztcV09-UMjYxPTc0IKGQDm8ESgAGYAThCdSGQgOBBIciBJMGBQSHxGz0Y43CUHocF6GW8dweSECLzeiCMNC+PyQALEN3uj0QcNeaOR31+iH+gOB7FB4MhGJA0OxSPhSCCBNRiOI1Np+IZiCZn0JfwBNCBYBBYIhv3ZWPeNDx3OZRJJgrJlgpYqhksQHxlPJRRIFICF4TgECB4u0HMQAFZpQiAGyK4Xk0VUs3qq1cgDsEphiDtXOeOv5pIdyqdpsx3t9Mv9fNZ9sNxqwYZp6vd1thcsDccdlKT5tTXLhAdj+qVICNJrV3ueMo+Rb1BvY5cTokoQiAA
\begin{tikzcd}
  & A' \arrow[ld, dashed] \arrow[rd] &                                                                            \\
L_n = \mathsc{chain}(L) \arrow[d, dashed, bend right] \arrow[d, bend left] &                                  & R_n = \mathsc{chain}(L) \arrow[d, dashed, bend right] \arrow[d, bend left] \\
L_{n-1} \arrow[d, dashed, bend right] \arrow[d, bend left]                 &                                  & R_{n-1} \arrow[d, bend left] \arrow[d, dashed, bend right]                 \\
... \arrow[rd, dashed, bend right] \arrow[rd, bend left]                   &                                  & ... \arrow[ld, bend left] \arrow[ld, dashed, bend right]                   \\
  & L_0 = \boxed{1} = R_0            &                                                                           
\end{tikzcd}
\end{center}

Now, just like closing a zipper, we will merge the AQDDs $L_{n-1}$ and $R_{i}$ to get $A_i$ and then merge $A_i$ with $L_{i+1}$ and $R_{i+1}$ to get $A_{i+1}$. We will repeat this process until we get $A_n = 1$. This process is guaranteed to terminate because the height of the AQDDs we are merging is increasing at each step and is lower than $n$.

\end{proof}

\section{Abstract additive quantum decision diagrams}

\subsection{One-side case}

To begin, let's consider the simple case where all diagrams only have left children (this case would be useless in practice because it would result in only one interval and zeros). Let's say we want to merge the two diagrams:
\begin{gather*}
A = (\{(\alpha_0, a_0), ..., (\alpha_{l-1}, a_{l-1}), (\beta(A)_0, b_0), ..., (\beta(A)_{n-1}, b_{n-1})\}, \emptyset)\quad \text{and} \\
C = (\{ (\beta(C)_0, b_0), ..., (\beta(C)_{n-1}, b_{n-1}), (\gamma_0, c_0), ..., (\gamma_m, c_{m-1}) \}, \emptyset)
\end{gather*}
\noindent with $\{a_0, ..., a_{l-1}\} \cap \{c_0, ..., c_{m-1}\} = \emptyset$. A graphic representation of our merging formula would be:

\vspace{1.7cm}
\leftskip=200pt
% https://tikzcd.yichuanshen.de/#N4Igdg9gJgpgziAXAbVABwnAlgFyxMJZAJgBoAGAXVJADcBDAGwFcYkQQBfU9TXfQigAsFanSat2XHiAzY8BIgHZRNBizaIO3XvIFEAnKvEapO2XwWCSpAIxj1krQEFpu-ouF2HEzSADCbhZ6nsgArN5qvmYych7WKvZRplpBcVZEAByRJk4gADr5ODAAHjjAALYwAE4A5jCcABTOpAAE-gCUaZb6KOSkxD4pIPQA+uTdIda2A0N5Y8CMALS2nJPxRGSDyXkARuPrGSgAzLM7fvvAYCtr5um9yCLbuX4Axgd3PaERz45vo5UbocHgA2M4vdhjCafKbKcF-SEA5arYGhbK-aJafbQ2JfaxGDHDS7XFEwjYoWz9Ql5d449xHZC2GbU-6A0liGBQeoIFCgABm1QgFSQ-RAOAgSBEEK0zBANEY9F2MEYAAU8exGDA+TgggKhUgZmKJYgItK6HKQAqlar1VpNdrdYLhYgyEakNkzbL5YrlWrYXatTrzHrnac3YgPQitLQLVbfbaQNUsLUABZBmQhyU0cUG0VRsDMRiMb3Wv3ky2Bx36xBSnOIJnnJAFosl+P+isO4NOrPh2yuqMFfJKnD0ZodUYAK1jPpt7ft6f53Zr2eNRjNhSYaBTYyw09LCaTqYXIEzJpXBsN+cLxctM7LRw7x9PprrfcbWkKw9HACFx1PW7O5bzlWzovsathhleLa3vuc6Vl21ZgQaUoDoUtT0BUFRjAA1nubZAfBGZLmC4YqGajR8hUXQAfevSPrGWBgH4UAQMwuyaiB7rnvWJFQTecaAQ+wEIc6HqvmRqH5LAjAjpOeGCXRwlEdWYngZGmLNvxd4Jkpi4qdxtimpJm7bqMu40QeyZppxEYGWufEWXBnbKaJ3HEHmmKDuhmE4fJtGePRIlIGur4eSkmmOQRDo0CmMD0FA7CQExDHJVoUD0HAsUJUF9aGq+-YadekVCZWMVxQlWhJWw8qMcxGVZTZEEGShhXQQJ-mCPRZXxYlBDVZatXsOlmWco1SE8e+EUwfhJXRSAsU9ZVfUpXVI3ZS5Bpka+6nhUV00KQFwHdRV4DLTVqUgMNDU5bYIXGu5k17e1OmlfN5W9cl52rddlCcEAA
\begin{tikzcd}[overlay]
                               &         & {} \arrow[d, "u"]                                                        &         & {} \arrow[d, "v"]                                                       &                                   &                                & {} \arrow[rd, "u"] &                                                                                                                                & {} \arrow[ld, "v"'] &                                &         \\
                               &         & A \arrow[ld] \arrow[d] \arrow[rd, "\beta(A)_j"] \arrow[lld, "\alpha_i"'] &         & C \arrow[lld] \arrow[ld, "\beta(B)_j"] \arrow[d] \arrow[rd, "\gamma_k"] & {} \arrow[rr, "(\text{fm})", Rightarrow] &                                & {}                 & {\text{fm}(A, C)} \arrow[ld] \arrow[d, "\delta_j"] \arrow[rd] \arrow[lld, "\alpha_i"'] \arrow[rrd] \arrow[rrrd, "\gamma_k"] &                     &                                &         \\
a_0 \arrow[r, no head, dashed] & a_{l-1} & b_0 \arrow[r, no head, dashed]                                           & b_{n-1} & c_0 \arrow[r, no head, dashed]                                          & c_{m-1}                           & a_0 \arrow[r, no head, dashed] & a_{l-1}            & b_0 \arrow[r, no head, dashed]                                                                                                 & b_{n-1}             & c_0 \arrow[r, no head, dashed] & c_{m-1}
\end{tikzcd}

\vspace{1.7cm}
\leftskip=0pt

\noindent with $\delta_j = \beta(A)_j \sqcup \beta(B)_j$. More formally, the merging formula would be:
$$\text{fm}(A, C) = (\{(\alpha_0, a_0), ..., (\alpha_{l-1}, a_{l-1}), (\delta_0, b_0), ..., (\delta_{n-1}, b_{n-1}), (\gamma_0, c_0), ..., (\gamma_m, c_{m-1})\}, \emptyset)$$

\subsection{Fully connected case}


Let's consider another case, where our two nodes $A$ and $B$ have both a common left-descendance and a common right-descendance. We will see later that the general case can always be reduced to this case. The abstract amplitude on the link between two nodes is $\text{ampl}(u, v)$, for example $\text{ampl}(A, l_0)$ or $\text{ampl}(B, l_0)$. Additionally, let $X = \{l_0, ..., l_{k-1}, r_0, ..., r_{m-1}\}$.

\vspace{1.7cm}
\leftskip=200pt
% https://tikzcd.yichuanshen.de/#N4Igdg9gJgpgziAXAbVABwnAlgFyxMJZAJgBoAGAXVJADcBDAGwFcYkQBBEAX1PU1z5CKACwVqdJq3YAhHnxAZseAkXKliEhizaIQjAPrl5-ZUKJlNNbdL2HgYALQBGbicUCVw5GKuSd7ABORu5KgqooAGwaWlK6IMHAALYubrym4d4A7DHWceyGxukeZhHIAJy5-rb6Bg6poZ7mKM7OVTbxwUUKYV5EzgDM7fl6iSmujaXezupUeQF6AMIABAC8ywA6GzgwAB44wABmSdwAFBykyzIAlJOZRNHOsQsgd30oABykT-M1PBIwKAAc3gRFAh0CECSSHUIBwECQZH0WDA8Sg9DgAAtAe4IVCYTR4UghsjUex0ViccU8dDEEiiYgSYwUWiIDgdlAQDRsfROXpIGTqZDaW04Qi6TRmWS9BTsZyhfjEKKGUyWeSMXLccKCWKkCIFSLCeL9QoaTqGQBWA16o1IC3cmC89gCtiStUytkcrWK5Xiq2m7WIaK6xA5Ums9k4h1O-kENjWxBfEOVEA8vngONc8Pkz1UgM+2EMsNStEavPgwMzW2JhNV5O131IZxIks55gAI0YrtTjr5YGYjEYhPoWEYzszv3iWx2+2Ap2O1zS+ZFhfFwdbMrL8so3CAA
\begin{tikzcd}[overlay]
                                &  & A \arrow[lldd, dashed] \arrow[dd, dashed] \arrow[rrdd] \arrow[rrrrdd] &  & B \arrow[lllldd, dashed] \arrow[lldd, dashed] \arrow[dd] \arrow[rrdd] &  &                                          &                                 &    &         & {C = \text{fm}(A, B)} \arrow[ldd, dashed] \arrow[rdd] \arrow[rrrdd] \arrow[llldd, dashed] &                                 &  &         \\
                                &  &                                                                       &  &                                                                       &  & {} \arrow[rr, "\text{(fm)}", Rightarrow] &                                 & {} &         &                                                                                           &                                 &  &         \\
l_0 \arrow[rr, no head, dotted] &  & l_{k-1}                                                               &  & r_0 \arrow[rr, no head, dotted]                                       &  & r_{m-1}                                  & l_0 \arrow[rr, no head, dotted] &    & l_{k-1} &                                                                                           & r_0 \arrow[rr, no head, dotted] &  & r_{m-1}
\end{tikzcd}

\vspace{1.7cm}
\leftskip=0pt

Where the new abstract amplitudes are defined by the following formula:

$$\forall x \in X, \text{ampl}(C, x) = \text{ampl}(A, x) \sqcup \text{ampl}(B, x)$$

\noindent\underline{\textbf{Proof:}} Let $f : \mathcal{D}_n \times \mathcal{D}_n \rightarrow \mathcal{D}_n$ be our diagram transofrmation.
Let $\alpha_0, \alpha_1, \beta_0, \beta_1 \in \mathcal{A}_0$ such that $\alpha_0 \subset \beta_0$ and $\alpha_1 \subset \beta_1$.
\begin{align*}
&\min(\min \RE(\beta_0), \min \RE(\beta_1)) \le \min(\min \RE(\alpha_0), \min \RE(\alpha_1)) & \text{and} \\
&\max(\max \RE(\beta_0), \max \RE(\beta_1)) \ge \max(\max \RE(\alpha_0), \max \RE(\alpha_1))&
\end{align*}

\noindent thus $\RE(\alpha_0 \sqcup \alpha_1) \subset \RE(\beta_0 \sqcup \beta_1)$. The same goes for imaginary parts, hence $\alpha_0 \sqcup \alpha_1 \subset \beta_0 \sqcup \beta_1$.
With a very similar proof, we can show that $\alpha_0 + \alpha_1 \subset \beta_0 + \beta_1$.

From there it comes that $\forall x \in X, \text{ampl}(A, x) \subset \text{ampl}(C, x)$ and $\forall x \in X, \text{ampl}(B, x) \subset \text{ampl}(C, x)$, and inductively:
\begin{align*}
\displaystyle\sum_{l \in \{l_0, ..., l_{k-1}\}} \text{ampl}(A, l) * l \subset &\displaystyle\sum_{l \in \{l_0, ..., l_{k-1}\}} \text{ampl}(C, l) * l\quad\quad \text{and} \\
\displaystyle\sum_{r \in \{r_0, ..., r_{m-1}\}} \text{ampl}(A, r) * r\subset &\displaystyle\sum_{r \in \{r_0, ..., r_{m-1}\}} \text{ampl}(C, r) * r
\end{align*}

We now have $A \le C$, and since $A$ and $B$ are interchangeable, $B \le C$. Consequently, $f$ is a merge approximation and according to the merging theorem, a global approximation can be derived from it.
\hfill{} $\boxed{}$

