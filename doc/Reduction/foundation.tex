\subsection{Approximations}

We note that multiple QDDs can be evaluated to the same abstract state. This part will aim to provide an algorithm to decrease the "size" of diagrams while not breaking their evaluation by $\mathcal{E}$. The size of a AAQDD is the number of intervals (counting them with their multiplicity). We would want, from a diagram $D$, to get a diagram $D'$ such that $\text{size}(D) > \text{size}(D')$ and $D \le D'$ (the reduction is smaller in size and might be less precise than the original diagram). More generally, function $g : \mathcal{D}_n \rightarrow \mathcal{D}_n$ is an \textit{global approximation} if
$$\forall D \in \mathcal{D}_n, D \le g(D)$$

Similarly, a function $f : \mathcal{D}_n \times \mathcal{D}_n \rightarrow \mathcal{D}_n$ is a \textit{merge approximation at height $n$} if
$$\begin{cases}
    \forall A \not= B \in \mathcal{D}_n, A \le f(A, B)~\text{and}~B \le f(A, B) \\
    \forall A \in \mathcal{D}_n, f(A, A) = A
\end{cases}
$$

We developed a reduction formula to force the merging of two nodes. This is permitted both thanks to abstract interpretation (to merge without amplitudes being colinear) and the additive nature of diagrams. The cost $\mathcal{C}_f : \mathcal{D}_n \rightarrow \mathbb{R}^+$ of applying a global approximation $f$ to a diagram $D \in \mathcal{D}_n$ is defined by:

$$\mathcal{C}_f(D) = \mathcal{I}(f(D)) - \mathcal{I}(D)$$

\subsection{Merging theorem}

Let $n \in \mathbb{N}^*$, $N \ge n$, and $f$ be a merge approximation. Moreover let $w : \mathcal{D}_N \rightarrow \mathcal{D}_n \times \mathcal{D}_n$ be a choice function at height $n$ in $\mathcal{D}_N$, meaning that $\forall D, w(D) \in \mathcal{N}_n(D) \times \mathcal{N}_n(D)$. Now we define:

$$\function{f | w}{\mathcal{D}_N}{\mathcal{D}_N}{D}{r_N(B, C, r_N(A, C, D))\quad \text{with}~C = f(w(D))}$$

With $\forall i > 0, r_i : \mathcal{D}_n \times \mathcal{D}_n \times \mathcal{D}_i \rightarrow \mathcal{D}_i$ the replacement function defined by:
$$\begin{cases}
\forall n > i, \forall A, B \in \mathcal{D}_n, \forall D \in \mathcal{D}_i, r_i(A, B, D) = D \\
\forall A, B, D \in \mathcal{D}_n, r_n(A, B, D) = \begin{cases}
B \quad \text{if}~D = A \\
D \quad \text{otherwise}
\end{cases} \\
\forall 1 \le n < i, \forall A, B \in \mathcal{D}_n, \forall \{(\alpha_j, g_j)\}, \{(\beta_k, d_k)\} \in \mathcal{D}_{i-1},\\
\hfill r_i(A, B, (G, D)) = (\{(\alpha_j, r_{i-1}(g_j))\}, \{(\beta_k, r_{i-1}(d_k))\})
\end{cases}
$$

\noindent\underline{\textbf{Merging theorem:}} Let $f$ be a merge approximation at height $n$ and $w$ be a choice function at height $n$ in $\mathcal{D}_N$. $f|w$ is a global approximation in $\mathcal{D}_N$.

\noindent\underline{Proof:} Let $D \in \mathcal{D}_N$, $(A, B) = w(D)$ and $C = f(w(D))$. By ascending induction (in height), it comes that $\forall U, V \in \mathcal{D}_n, U \le V \Rightarrow r_N(U, V, D) \le D$. Meanwhile, $f$ is a merge approximation so $A \le C$, thus $r_N(A, C, D) \le D$.

Proving that $r_N(A, C, D) \le D$ is mostly enough to prove the theorem, because once it is proven we only have let $D' = r_N(A, C, D)$ and use this result on $B$ and $D'$ to conclude that $r_N(B, C, D') \le D'$. The only problem is that we would need $C' = f(w(D'))$ instead of $C$ to reuse the exact same result demonstrated earlier.
\hfill{} $\boxed{}$
