\subsection{Real intervals}

The standard definition of real intervals is:
$$\forall a, b \in \mathbb{R}, [a, b]
= \{x \in \mathbb{R} / \min(a, b) \le x \le \max(a, b)\}$$

From now on, the set of real intervals will be noted $\mathcal{A}_0(\mathbb{R})$. Of course, $\mathcal{A}_0(\mathbb{R}) \subset \mathcal{P}(\mathbb{R}) \subset \mathcal{P}(\mathbb{C})$. While not being sufficient to handle completely our operations on quantum states, caracterized by complex amplitudes, these real intervals will still be useful. Moreover, they are well-known and many papers already studied them.

\subsection{Closure \& operations}
\label{closure-operations}

Let $E \subset \mathcal{P}(\mathbb{C})$ a set containing sets of $\mathbb{C}$ (later on, our intervals).

\begin{definition}[closure]
    The closure $\mathscr{C}$ of a set $a \in \mathcal P (\mathbb C)$ is defined by

    $$\mathscr{C}(a) = \bigcap_{\gamma \supset a \mand \gamma \in \mathcal{A}_0(\mathbb{C})} \gamma$$

    Additionally, for any operator
    $\odot : {\mathcal{A}_0}^2 \rightarrow \mathcal{P}(\mathbb C)$,
    we define the operator
    $\mathscr C(\odot) : {\mathcal{A}_0}^2 \rightarrow \mathcal{A}_0$
    such that
    $$\forall a, b \in \mathcal P (\mathbb C),
    a ~ \mathscr C(\odot) ~ b = \mathscr C (a \odot b)$$
\end{definition}

\begin{prop}[colsure]
    The closure is well-defined.
    \begin{itemize}
        \item If $\odot$ is commutative, so is $\cdot$
        \item 
    \end{itemize}
\end{prop}

\begin{definition}[operations]
    Let $E \subset \mathcal{P}(\mathcal C)$. We define for $\alpha, \beta \in \mathcal{P}(\mathbb C)$ the following operations
    $$\alpha \otimes \beta = \{a b ; a \in \alpha, b \in \beta\}$$
    $$\alpha \oplus \beta = \{a + b ; a \in \alpha, b \in \beta\}$$

    Note that none of these sets are in $E$ in the general case, despite it being the case for $E = \mathcal{A}_0(\mathbb{R})$. We define the \textbf{sum} in $\mathcal{A}_0$ to be $+ = \mathscr{C}(\oplus)$ and the \textbf{product} in $\mathcal{A}_0$ to be $\cdot = \mathscr{C}(\oplus)$. Additionally, we define the \textbf{join} $\sqcup = \mathscr{C}(\cup)$.
\end{definition}

\noindent Remarkably, in + is the same as $\oplus$ and $\cdot$ is the same as $\otimes$. On complex intervals, we will see later that this is not necessarily the case.

\begin{definition}[modulus]
    For all $\alpha$ in $E$, we define when it exists
   $\forall \alpha \in E, |\alpha| = \sup\{|a| ; a \in \alpha\}$.
\end{definition}

This general case on a contextual set $E$ leads to defining a \textbf{partial order $\le$}, which really is the subset relation $\supset$, but implying that the sets are all intervals of the same interval set $E$. We easily get the following properties
\begin{enumerate}[i]
    \item $\forall \alpha_1, \beta_1, \alpha_2, \beta_2 \in \mathcal{A}_0, \alpha_1 \le \alpha_2 ~\text{and}~ \beta_1 \le \beta_2 \Rightarrow \alpha_1 + \alpha_2 \le \beta_1 + \beta_2$
    \item $\forall \alpha_1, \beta_1, \alpha_2, \beta_2 \in \mathcal{A}_0, \alpha_1 \le \alpha_2 ~\text{and}~ \beta_1 \le \beta_2 \Rightarrow \alpha_1 \alpha_2 \le \beta_1 \beta_2$
\end{enumerate}

Note that in the case of $\mathcal{A}_0(\mathbb{R})$, the sum and product are \textbf{direct}, meaning that for all $\alpha, \beta \in \mathcal{A}_0^+, \alpha + \beta = \alpha \oplus \beta$ and $\alpha \beta = \alpha \otimes \beta$. To prove that the sum (respectively, the product) is direct, it is enough to prove that summing (respectively multiplying) two elements of $E$ yields another element of $E$.

The operations $\oplus, \otimes, \sqcup, +$ and $\cdot$ are obviously commutative, but only $\oplus$ and $\otimes$ are always associative no matter the set $E$. Proving the directness of the sum or the product in $E$ is hence a way to prove their associativity in $E$.

\subsection{Remarkable subsets}

Intervals of $\mathcal{A}_0(\mathbb{R})$, while being already unsifficient to fully cover the use of intervals in a quantum context, contain even more resricted subsets that will be useful in the next sections. The set of positive intervals $\mathcal{A}_0^+ = \mathcal{A}_0(\mathbb{R}) \cap \mathcal{P}(\mathbb{R})$ is remarkably convenient because it is working well with the operations defined above.

First, $\mathcal{A}_0^+$ is stable for +, $\cdot$ and $\sqcup$, meaning that using these operations on two elements of $\mathcal{A}_0^+$ results in another element of $\mathcal{A}_0^+$. Second, for all real number $0 \le a_1 \le b_1$ and $0 \le a_2 \le b_2$
$$[a_1, b_1] + [a_2, b_2] = [a_1 + a_2, b_1 + b_2]$$
$$[a_1, b_1] [a_2, b_2] = [a_1 a_2, b_1 b_2]$$

Let $\mathcal{A}_0^{2\pi} = \mathcal{A}_0^+ \cap \mathcal{P}([0, 2\pi])$. While $\mathcal{A}_0^{2\pi}$ is not stable, it will be useful in section \ref{polar}.

\begin{theorem}[intervals of the exponential]
Let $\theta \in \mathcal{A}_0^+$, there is a unique $\varphi \in \mathcal{A}_0^{2\pi}$ such that
$$\exp(i\theta) = \exp(i \varphi)$$
\end{theorem}

\begin{proof}
There are unique $\theta^-, \theta^+, \mathbb{R}^+$ such that $\theta = [\theta^-, \theta^+]$ and $\theta^- \le \theta^+$.
% TODO: change the definition?

\end{proof}
