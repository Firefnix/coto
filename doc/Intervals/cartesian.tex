\subsection{Definition}

Real intervals can be generalised to complex intervals naturally using the cartesian notation of complex numbers.
$$\forall x, y \in \mathbb{C}, [x, y]
= \{a + ib ; a \in [\RE(x), \RE(y)], b \in [\IM(x), \IM(y)]\}$$

Similarly to what can be done with real intervals, we can see complex numbers as complex intervals that only have one element, hence we might use $\mathbb{C}$ to be $\{[z, z] ; z \in \mathbb{C}\}$. Now let $\mathcal{A}_0 = \{[x, y] ; x, y \in \mathbb{C}\}$.

\subsection{Operations}

The basic operations on $\mathcal{A}_0$ (sum, product and join) are defined thanks to subsection \ref{general-operations}. Note that the product is \underline{not} associative in $\mathcal{A}_0$, and that the sum is \underline{not} distributive on the product:
$$\forall \alpha, \beta, \gamma \in \mathcal{A}_0, (\alpha + \beta) \gamma \subset \alpha \gamma + \beta \gamma$$

\noindent \underline{\textbf{Proof:}}
\begin{align*}
(\alpha + \beta)\gamma &= \{xc ; x \in \{a + b ; a \in \alpha, b \in \beta\}, c \in \gamma\} \quad \text{(by definition of the product)}\\
&= \{(a + b)c ; a \in \alpha, b \in \beta, c \in \gamma\} \\
&= \{ac + bc ; a \in \alpha, b \in \beta, c \in \gamma\} \\
&\subset \{ac + bd ; a \in \alpha, b \in \beta, c, d \in \gamma\} \\
&\subset \{ac; a \in \alpha, c \in \gamma\} + \{bc ; b \in \beta, c \in \gamma\} \quad \text{(by definition of the sum)} \\
&\subset \alpha \gamma + \beta \gamma
\end{align*}
\hfill{} $\boxed{}$

\subsection{Convex conservation}

All complex intervals are convex.

\noindent \underline{\textbf{Proof:}} Let $\alpha \in \mathcal{A}_0$ and $a, b \in \alpha$. Additionally let $t \in [0, 1]$.

\begin{align*}
\RE(ta+(1-t)b) &= \RE(ta) + \RE((1-t)b) \\
&\le t\RE(a) + (1-t)\RE(b) \quad (\text{since}~t, 1-t \in \mathbb{R}) \\
&\le \max(\RE(\alpha)) \quad \text{since}~a, b \in \alpha
\end{align*}

The corresponding properties with a minimum or the imaginary part are proven very similarly, hence $ta+(1-t)b \in \alpha$.
\hfill{} $\boxed{}$

\subsection{The minus operation}

This gives \textit{almost} us an ring structure for $(\mathcal{A}_0, +, \cdot)$, which we will use just next, and form now on we will note $0 = [0, 0]$ and $1 = [1, 1]$. However it is indeed not a ring because $(\mathcal{A}_0, +)$ is not a group. In fact, the exitence of an opposite (additive inverse) in $\mathcal{A}_0$ is replaced by the following property, which implies that almost no complex interval has an opposite:
$$\forall \alpha,\beta \in \mathcal{A}_0, \alpha \not\in \mathbb{C} \Rightarrow \alpha + \beta \not= 0$$

\noindent \underline{\textbf{Proof:}} Let $\alpha, \beta \in \mathcal{A}_0$, such that $\alpha \not\in \mathbb{C}$. Let $b \in \beta$.
Now, since $\alpha \not\in \mathbb{C}$ there are at least two different complex numbers $x, y \in \mathbb{C}$ in $\alpha$. Hence one of them $z \in \{x, y\}$ is different from $-b$, so $z + b \in \alpha + \beta ~\text{and}~ z + b \not=0$ and finally $\alpha + \beta \not= 0$.
\hfill{} $\boxed{}$

\vspace{1em}
This implies the non-existence of an additive inverse for all non-constant intervals, which breaks any ring structure we could try to build on $+$. Despite that, we can still define a minus operation
$$\forall \beta \in \mathcal{A}_0, -\beta = \{-b ; b \in \beta\}$$
$$\forall \alpha, \beta \in \mathcal{A}_0, \alpha - \beta = \alpha + (- \beta)$$

What is important keep in mind is that we cannot do things like "$\alpha + \beta = \gamma$ \textit{so} $\alpha = \gamma - \beta$", for example if $\alpha = \beta = [0, 1]$. Fundamentally, $\alpha - \alpha \not= 0$.

\subsection{Partial order on intervals}

We define the relation $\le$ on $\mathcal{A}_0$ such that $\forall \alpha, \beta \in \mathcal{A}_0, \alpha \le \beta \Rightarrow \alpha \subset \beta$.
We easily get the following properties:
\begin{enumerate}[i]
    \item $\forall \alpha, \beta \in \mathcal{A}_0, \alpha \le \beta \Rightarrow \alpha^c \le \beta^c$
    \item $\forall \alpha, \beta, \gamma \in \mathcal{A}_0, \alpha \gamma \le \beta \gamma ~\text{and}~ \gamma \not=0 \Rightarrow \alpha \le \beta$
\end{enumerate}

\subsection{Centering}

For all $\alpha \in \mathcal{A}_0$, we introduce $\mu(\alpha) \in \mathbb{C}$ the \textit{middle} of $\alpha$:
$$\mu(\alpha) = \frac{\min(\RE~\alpha) + \max(\RE~\alpha)}{2} + i \frac{\min(\IM~\alpha) + \max(\IM~\alpha)}{2}$$
This notion comes with the following properties:
\begin{enumerate}[i]
    \item $\forall x \in \mathbb{C}, \mu([x, x]) = x$
    \item\label{musum} $\forall \alpha, \beta \in \mathcal{A}_0, \mu(\alpha + \beta) = \mu(\alpha) + \mu(\beta)$
    \item \label{muscalarprod}$\forall \alpha \in \mathcal{A}_0, \forall \lambda \in \mathbb{C}, \mu(\lambda \alpha) = \lambda \mu(\alpha)$ and more generally,
    \item $\forall \alpha, \beta \in \mathcal{A}_0, \mu(\alpha \beta) = \mu(\alpha) \mu(\beta)$
\end{enumerate}

\noindent\underline{\textbf{Proofs:}}
\begin{enumerate}[i]
    \item $\RE([x, x]) = \RE(x)$ and $\IM([x, x]) = \IM(x)$, hence the result.
    \item $\forall \alpha, \beta \in \mathcal{A}_0, \RE(\alpha + \beta) = \RE(\alpha) + \RE(\beta)$
    \item The result is obvious for $\lambda \in \mathbb{R}$.
    Moreover it is clear that $\IM(i \alpha) = \RE(\alpha)$ and similarly that $\RE(i \alpha) = - \IM(\alpha)$, thus
    \begin{align*}
        \mu(i\alpha) &= \frac{\min(-\IM~\alpha) + \max(-\IM~\alpha)}{2} + i \frac{\min(\RE~\alpha) + \max(\RE~\alpha)}{2} \\
        &= \frac{(-\max(\IM~\alpha)) + (-\min(\IM~\alpha))}{2} + i \frac{\min(\RE~\alpha) + \max(\RE~\alpha)}{2} \\
        &= - \IM(\mu(\alpha)) + i \RE(\mu(\alpha)) \\
        &= i \mu(\alpha)
    \end{align*}
    Property \ref{musum} enables us to conclude.
    \item The case when $\alpha$ is a real interval is easy. Using property \ref{muscalarprod} and the almost-ring structure of $\mathcal{A}_0$, the general case is proven.
    \hfill{} $\boxed{}$
\end{enumerate}

\noindent From these properties, it follows that $\mu$ is an almost-ring morphism.

For all $z \in \mathbb{C}$, let $\pm z = [-z, z]$ (obviously $\mu(\pm z) = 0$).
\begin{theorem}[central decomposition]
    $$\forall \alpha \in \mathcal{A}_0, \exists z \in \mathbb{C}, \{y \in \mathbb{C} / \alpha = \pm y + \mu(\alpha)\} = \{z, -z, \bar{z}, -\bar z\}$$
\end{theorem}
\begin{proof}
    The existence on real intervals comes easily from the definition of $\mu$.
    Now for $\alpha \in \mathcal{A}_0$, $\alpha = \RE(\alpha) + i~\IM(\alpha)$, hence the existence of $z$ such that $\{z, -z, \bar{z}, -\bar z\} \subset \{y \in \mathbb{C} / \alpha = \pm y + \mu(\alpha)\}$.
    The reciprocal inclusion comes from the fact that, for real numbers, it is true.
\end{proof}

Now let $\alpha^c$ be this centered interval $\alpha - \mu(\alpha)$, and $\mathcal{A}_0^c = \text{Ker}(\mu)$ be the sub-almost-ring of centered intervals. Now that we defined the centered version of an interval, we can come back to computing $\alpha - \alpha$
$$\forall \alpha \in \mathcal{A}_0, \alpha - \alpha = 2(\alpha^c) = (2\alpha)^c$$

\begin{align*}
\alpha - \alpha
&= \mu(\alpha) + \alpha^c + (\mu(-\alpha) + (-\alpha)^c)\\
\end{align*}

\subsection{Magnitude}

For all $\alpha \in \mathcal{A}_0$, we introduce $|\alpha| \in \mathbb{R}$ the \textit{magnitude} of $\alpha$

$$|\alpha| = \max \{|x| ; x \in \alpha\}$$

\begin{prop}[magnitude from boundaries]
    $$\forall x, y \in \mathbb C, \left|[x, y]\right| = \max (|x|, |y|)$$
\end{prop}
\begin{proof}
    Let $x, y \in \mathbb C$. $|x| \in \{|z| ; z \in [x, y]\}$ so $|x| \le \left|[x, y]\right|$ and $|y| \le \left|[x, y]\right|$ thus $\max (|x|, |y|) \le \left|[x, y]\right|$.
    Now let $z \in [x, y]$. $|\RE(z)| \le \max (|\RE(x)|, |\RE(y)|)$ and equally for the imaginary part, hence $|z| \le \max (|x|, |y|)$ by the triangular inequality, so $|[x, y]| \le \max (|x|, |y|)$.
\end{proof}
