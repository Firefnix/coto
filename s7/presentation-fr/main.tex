\documentclass[french, 12pt]{beamer}

\usepackage[T1]{fontenc}
\usepackage[french]{babel}
\usepackage{url}

\DecimalMathComma
\usepackage{braket}

\usepackage{svg}
\svgsetup{inkscapelatex=false}

\usepackage{amsmath}
\usepackage{pgfplots}
\pgfplotsset{compat=1.18}
\usepackage{amsfonts}

\usepackage{xcolor}
\definecolor{ups}{RGB}{86,08,59}
\definecolor{cs}{RGB}{148,13,56}

\usepackage{xfrac}
\newcommand{\somme}{\displaystyle\sum}

\usetheme[compress]{Berlin}
\beamertemplatenavigationsymbolsempty
\setbeamercolor*{palette primary}{bg=cs, fg=white}
\setbeamercolor*{palette secondary}{bg=ups, fg=white}

\setbeamercolor*{enumerate item}{fg=cs}
\setbeamercolor*{enumerate subitem}{fg=ups}
\setbeamercolor*{enumerate subsubitem}{fg=cs}
\setbeamercolor*{itemize item}{fg=cs}
\setbeamercolor*{itemize subitem}{fg=ups}
\setbeamercolor*{itemize subsubitem}{fg=cs}

\setbeamertemplate{page number in head/foot}[framenumber]
\setbeamertemplate{frametitle}{}
\useoutertheme{split}

\title{Autour des diagrammes de décision quantiques}
\author{Malo Leroy}
\institute{Parcours recherche -- CentraleSupélec}

% Figures and diagrams
\usepackage{tikz}
\usetikzlibrary{positioning}
\usetikzlibrary{
    pgfplots.dateplot,
}
\usepackage{tikz-cd}

\begin{document}

\begin{frame}
    \titlepage
\end{frame}

% 1. Replacez votre projet dans son contexte : enjeux scientifiques, enjeux sociétaux et/ou économiques (le cas échéant)
\begin{frame}{Contexte}

\begin{center}
  \begin{columns}
    \begin{column}{0.5\textwidth}
      Les  bases de données croissent rapidement

      \vspace{1em}
      Les algorithmes classiques sont parfois inefficaces
    \end{column}
    \begin{column}{0.5\textwidth}
      \begin{tikzpicture}
        \begin{axis}[
          width=6cm,height=6cm,
          x tick label style={
            /pgf/number format/1000 sep=},
          xtick = {1998,2008},
          enlargelimits=0.05,
          legend style={at={(0.5,0.1)},
          anchor=north,legend columns=-1},
        ]
        \addplot
          coordinates {(1998,7.414973347970818) (2000,9)
             (2008,12)};
        \end{axis}
      \end{tikzpicture}
    \end{column}
    \let\thefootnote\relax\footnotetext{$\log_{10}$ du nombre de pages indexées par Google}
    % https://googleblog.blogspot.com/2008/07/we-knew-web-was-big.html
  \end{columns}


  \pause
  Les \textbf{algorithmes quantiques} permettent de résoudre certains problèmes plus efficacement
  \end{center}
\end{frame}

\begin{frame}{Le besoin}
  Les machines quantiques sont en développement et resteront coûteuses financièrement
  $$\Downarrow$$
  Il y a un besoin d'outils de simulation et de vérification d'algorithmes quantiques
\end{frame}

\begin{frame}{Simulations}
  Les simulations sont très coûteuses en temps de calcul
  \begin{table}[]
    \begin{tabular}{l|l|l|l}
        \textit{Grover} & Classique & Quantique & Simulation    \\ \hline \rule{0pt}{2.6ex}
    Complexité & $N$       & $\sqrt N$ & {\color{red}$N \sqrt N$}
    \end{tabular}
    % https://arxiv.org/pdf/2005.04635
  \end{table}

  \vspace{1em}

  Elles nécessitent une \textbf{structure de données} adaptée
\end{frame}

% État de l'art (ce qui a été déjà fait par d'autres)
\begin{frame}{État de l'art}
    \textbf{État de l'art}
    \begin{itemize}
        \item Interprétation abstraite
        \item Arithmétique des intervalles réels
        \item Diagrammes de décision quantiques
    \end{itemize}
    \pause
    \begin{center}
        Solution : \underline{diagrammes additifs abstraits}
    \end{center}
\end{frame}

\begin{frame}{Interprétation abstraite}
    L'\textbf{interprétation abstraite} permet de déterminer des propriétés ou d'accélérer des calculs

    \vspace{1em}
    \underline{Exemple :} signe d'une expression $e = (3 + 2) \times (-5)$
    \begin{align*}
        \text{signe}(e) &= (\text{signe}(3) + \text{signe}(2)) \times \text{signe}(-5) \\
        &= (\oplus + \oplus) \times \ominus \\
        &= \oplus \times \ominus \\
        &= \ominus
    \end{align*}
\end{frame}

\begin{frame}{Intérêts de l'interprétation abstraite}
    L'interprétation abstraite permet de déterminer des propriétés ou d'\textbf{accélérer des calculs}

    \vspace{1em}

    Elle peut être exacte ou \textbf{approximative}
\end{frame}

\begin{frame}
    \frametitle{Arithmétique des intervalles (1)}

    L'interprétation abstraite est applicable aux \textbf{intervalles réels}

    \begin{alignat*}{2}
    [1, 2] &* [-1, 1] &=& [-2, 2] \\
    [1, 2] &+ [-1, 1] &=& [0, 3] \\
    [1, 2] &\land [-1, 1] &=& [1, 1]
    \end{alignat*}

    \small{Le résultat de l'opération est \textbf{le plus petit intervalle} contenant tous les résultats élément par élément}
\end{frame}

\begin{frame}{Fonctions booléennes}

    Une \textbf{fonction booléenne}

    $$f : \{0, 1\}^n \to \{0, 1\}$$

    peut être représentée par une \textbf{table de vérité}

    \begin{center}
        \begin{tabular}{c|c|c|c}
            $x_1$ & $x_2$ & $x_3$ & $f(x_1, x_2)$ \\
            \hline
            0 & 0 & 0 & 0 \\
            0 & 0 & 1 & 0 \\
            0 & 1 & 0 & 0 \\
            0 & 1 & 1 & 1 \\
            1 & 0 & 0 & 1 \\
            1 & 0 & 1 & 1 \\
            1 & 1 & 0 & 1 \\
            1 & 1 & 1 & 1
        \end{tabular}

        \vspace{1em}
        \small{pour $f(x_1, x_2) = x_1 \lor (x_2 \land x_3)$}
    \end{center}
\end{frame}

\begin{frame}{Diagrammes de décision}
    Les \textbf{diagrammes de décision} permettent de représenter des fonctions booléennes

\vspace{1em}
% https://tikzcd.yichuanshen.de/#N4Igdg9gJgpgziAXAbVABwnAlgFyxMJZAJgBoAGAXVJADcBDAGwFcYkQAPAfQEYACADoDGEAE58AFN2KDh9MFD7cAzAEoQAX1LpMufIRTLSy6nSat2PTdpAZseAkXIVTDFm0QgpvdVp339J1IeV3MPL2lfGzs9RxRnYlD3dm81a39Yg2QeYySLT3J0210HLJyQmjd8zi4ZIUZ5RRUimNKiMkTKsPZmjVMYKABzeCJQADNRCABbJGcQHAgkADYaRiwwcKh6OAALAaKJ6dmaBaQckAAjGAUkZTmq8O5+AF4+Kz8QQ5nEFfnFxAArKt1pttnsoCAuslPNJnoVVvQrowAAolQKeURYQY7HAHSbfX6nRAAdih1Vh7xsXyQpL+SCBIDWG3YW12+zJjy4yj4r0KH2pJJO-3ODx6XOe70oGiAA
\begin{tikzcd}[ampersand replacement=\&, column sep=small]
    (x_1) \&                                                                \& x_1 \lor (x_2 \land x_3) \arrow[ld, dashed] \arrow[rddd, "x_1 = 1", bend left] \&   \\
    (x_2) \& x_2 \land x_3 \arrow[dd, "x_2=0"', dashed] \arrow[rd, "x_2=1"] \&                                                                                \&   \\
    (x_3) \&                                                                \& x_3 \arrow[ld, "x_3 = 0", dashed] \arrow[rd, "x_3=1"]                          \&   \\
            \& 0                                                              \&                                                                                \& 1
    \end{tikzcd}
\end{frame}

\begin{frame}
    Les \textbf{diagrammes de décision} permettent de représenter des fonctions booléennes

    \vspace{1em}
    \begin{center}
    % https://tikzcd.yichuanshen.de/#N4Igdg9gJgpgziAXAbVABwnAlgFyxMJZARgBoAGAXVJADcBDAGwFcYkQQBfU9TXfQigBMpAMzU6TVu2JceIDNjwEi5MRIYs2iEOTm8lA1aWIap2jtwP8VKMkLNb2XCTCgBzeEVAAzAE4QALZIaiA4EEiiNIxYYBZQ9HAAFm76IP5BITThSGQgAEYwYFCR5FbpAcGIUWERiCIgMXHsCcmp0fSFjAAKfMqCIH5Y7kk4aRlVNTmIACzlE0gz2XUNTfGJKSXzlYvLuZyUnEA
    \begin{tikzcd}[ampersand replacement=\&, column sep=huge, row sep=large]
        \& {} \arrow[ld, dashed] \arrow[rddd, bend left] \&   \\
    {} \arrow[dd, dashed] \arrow[rd] \&                                               \&   \\
        \& {} \arrow[ld, dashed] \arrow[rd]              \&   \\
    0                                \&                                               \& 1
    \end{tikzcd}

    \vspace{1em}

    On tire parti de la \textbf{structure} de la fonction
    \end{center}
\end{frame}

\begin{frame}{États quantiques}

    Un \textbf{état quantique} est une superposition d'états incompatibles

    $$\ket{\psi} = \alpha \ket{0} + \beta \ket{1} \quad \text{(un qubit)}$$

    \pause
    \begin{center}
        $n$ qubits $\Rightarrow$ $2^n$ états incompatibles
    \end{center}
On note les états sous forme de \textbf{vecteurs}
$$\alpha \ket {01} + \beta \ket {10} = \begin{pmatrix}
    \alpha \\ 0 \\ \beta \\ 0
\end{pmatrix}$$
\end{frame}

\begin{frame}{États quantiques (bis)}
    La représentation usuelle est proche des \textbf{tables de vérité}
    \begin{center}
        \begin{tabular}{c|c|c}
            $x_1$ & $x_2$ & $\braket{x_1 x_2|\psi}$ \\
            \hline
            0 & 0 & $\alpha$ \\
            0 & 1 & 0        \\
            1 & 0 & $\beta$  \\
            1 & 1 & 0
        \end{tabular}

        \vspace{1em}
        \small{pour $\ket \psi = \alpha \ket {00} + \beta \ket {10}$}
    \end{center}
\end{frame}

\begin{frame}{Diagrammes de décision quantiques}
    \begin{columns}
        \begin{column}{0.7\textwidth}
            Les états peuvent être représentés par des \textbf{diagrammes de décision quantiques}

            \vspace{1em}

            On tire parti de la \textbf{structure} de l'état
        \end{column}
        \begin{column}{0.3\textwidth}
            % https://tikzcd.yichuanshen.de/#N4Igdg9gJgpgziAXAbVABwnAlgFyxMJZABgBoAmAXVJADcBDAGwFcYkQAdDgIxgHMsYYGgC29HACcsADwC+ARgAEXLouJcYYKMLGSZskLNLpMufIRRkAzNTpNW7LtwjSYURfMPGQGbHgJE8qQ2NAwsbIggXiZ+5kRkxLZhDpFO-II64lJy5MoqanmqWIUcahpamXpyhrZufPBEoABmEhAiSGQgOBBIQSC8WkgAtFadjIIRIFD0cAAWbtEgLW0dNN1I5DTz9FDskGBsRs2t7YhWaz2InQO7Z53Jk1iLy6fnXZfXmrcj9-aT5CAaIx6LxGAAFUz+CwgKR8WY4QEgcYHdjTOYLWSUWRAA
\begin{tikzcd}[ampersand replacement=\&, column sep=small]
    \begin{pmatrix}2 \\ 0 \\ i \\ 0\end{pmatrix} \arrow[dd, "i", bend left] \arrow[dd, "2"', dashed, bend right] \&    \\
                                                                                                                    \&    \\
    \begin{pmatrix}1 \\ 0\end{pmatrix} \arrow[d, dashed, "1"', bend right]                       \&    \\
    \boxed 1                                                                                                     \& {}
    \end{tikzcd}
        \end{column}
    \end{columns}
\end{frame}

\begin{frame}{Diagrammes de décision quantiques}
    \begin{columns}
        \begin{column}{0.7\textwidth}
            Les états peuvent être représentés par des \textbf{diagrammes de décision quantiques}

            \vspace{1em}

            Dans le pire cela reste \textbf{exponentiel}
        \end{column}
    \begin{column}{0.3\textwidth}
    % https://tikzcd.yichuanshen.de/#N4Igdg9gJgpgziAXAbVABwnAlgFyxMJZABgBoAmAXVJADcBDAGwFcYkQQBfU9TXfQijIBmanSat2AHSkAjCAA8YUAAQBGLjxAZseAkTLExDFm0QdOY5QHN4RUADMAThAC2SMiBwQkamrJgwKCQAWmFPRiwwMxAoejgAC2VNRxd3RHIabw9-QODEcJoTSXMsFJBnNyRMrx9ETwCg0MLxU3ZyEBpGegDGAAU+PUEQJyxrBJxOkEjo9jjE5MtOIA
    \begin{tikzcd}[column sep=huge, row sep=large]
        {} \arrow[dd, "i", bend left] \arrow[dd, "2"', dashed, bend right] \\
\\
        {} \arrow[d, dashed, bend right]\\
        \boxed 1
        \end{tikzcd}
        \end{column}
    \end{columns}
\end{frame}

\begin{frame}{État de l'art}
    \textbf{Retour sur l'état de l'art}
    \begin{itemize}
        \item[\checkmark] Interprétation abstraite
        \item[\checkmark] Arithmétique des intervalles réels
        \item[\checkmark] Diagrammes de décision quantiques
    \end{itemize}
    \begin{center}
        On va utiliser ces concepts \underline{ensemble},

        avec une nouveauté : l'\textbf{additivité}
\vfill
% https://tikzcd.yichuanshen.de/#N4Igdg9gJgpgziAXAbVABwnAlgFyxMJZARgBoAGAXVJADcBDAGwFcYkQAdDgBWxAF9S6TLnyEU5UsWp0mrdlzQALLAH1iAoSAzY8BImWk0GLNok4dlagEybhusUWtSZJ+ecV9B90fpQBmClc5MwsAGRgAMxwAJywAcyUcehiYiAB3O20RPXFkABYXYxCFSxV1AGpFcttvbIc-ZAA2ItlTUrQvLR1fPIBWIOL2jx4vGRgoePgiUEi0gFskSRAcCCQyEEYsMFCoejglCay5iEXEZdWkZ03t3f3DqGOFpZpLxH86k7OAdle1xAGNx27D2ByOn2eiF+K3+TX4lH4QA
\begin{tikzcd}[ampersand replacement=\&, column sep=small]
    \& \Psi \arrow[ld, dashed] \arrow[d, dashed] \arrow[rd] \&      \& \Leftrightarrow \&               \& \Psi \arrow[ld, dashed] \arrow[rd] \&      \\
\phi_1 \& \phi_2                                               \& \psi \&                 \& \phi_1+\phi_2 \&                                    \& \psi
\end{tikzcd}
\end{center}
\end{frame}

\begin{frame}{Solution proposée}
    \textbf{Retour sur l'état de l'art}
    \begin{itemize}
        \item[\checkmark] Interprétation abstraite
        \item[\checkmark] Arithmétique des intervalles réels
        \item[\checkmark] Diagrammes de décision quantiques
        \item[+] Nouveauté : additivité
    \end{itemize}

    \begin{center}
        Solution : \underline{diagrammes additifs abstraits}
    \end{center}
\end{frame}


% 2. Présentez les objectifs de votre projet.
\begin{frame}{Objectifs}
    Objectifs
    \begin{itemize}
        \item \textbf{Modèle formel} de diagrammes de décision additifs abstraits
        \item \textbf{Implémentation} du modèle
    \end{itemize}
\end{frame}

% 4. Développez la méthodologie que vous avez mise en œuvre durant ces premiers mois, en la justifiant.
\begin{frame}{Méthodologie de travail}
\underline{\textbf{Méthodologie}}
\small{
\begin{center}
% https://tikzcd.yichuanshen.de/#N4Igdg9gJgpgziAXAbVABwnAlgFyxMJZABgBpiBdUkANwEMAbAVxiRBAF9T1Nd9CUARnJVajFmwA6knDAAeOYABEIAYyYBbGGBx08BDp24gM2fQOQBmEdXrNWiENNkLgAUTkwNaBvENceM34iACZSENE7CUdneUUAFQALAEuIACcsGH9jUz4CIgAWcMjxBycZOOAASW8GZK0dPX5-URgoAHN4IlAAMzSIDSQyEBwIJGExeyQwJgYGagY6ACMYBgAFXnM2Xx6cI17+wcQJ0aQwyejyl0VVOjgAAiY4FgY4QwXl1Y2g-McdvYCID6A3G1FOiGsFzKsVcyV0OHusHuDAA5HQ0jh3iBFit1ptgo4Mu1EgDjMCjudwZCotCKq4NNAABe+LE4r7434gIkkkDUFZgKBISzEQHkoVgsaIIpQqR0m7QLK87GfPE-ATYmC7fZAw7ikaS8406azebK3HfPLq-5K-mCiEism6xCUyXSo0xOXAWRwHBvJVs1WW7aagEUDhAA
\begin{tikzcd}[ampersand replacement=\&, column sep=small]
    {} \arrow[r] \& \text{Documentation} \arrow[rr, "\text{cas usuels}"] \arrow[rdd, "\text{état de l'art}"'] \&                                                                 \& \text{Exemples} \arrow[ldd, "\text{modèle}"', bend left] \arrow[rdd, "\text{tests}"] \&                       \\
                 \&                                                                                           \&                                                                 \&                                                                                      \&                       \\
                 \&                                                                                           \& \text{Théorie} \arrow[rr, "\text{code}"] \arrow[ruu, bend left] \&                                                                                      \& \text{Code}
    \end{tikzcd}
\end{center}
}
\end{frame}

\begin{frame}{Méthodologie (GitHub)}
    GitHub pour la gestion de projet
    \begin{itemize}
        \item \textbf{Issues} pour les tâches et le bugs
        \item Priorités, tailles, et deadlines
        \item Branches et \textbf{merge requests}
    \end{itemize}
\end{frame}

% 5. Présentez votre travail et vos éventuels résultats.
\begin{frame}{Travail réalisé}
  \textbf{Modèle}
  \begin{itemize}
      \item[\checkmark] Intervalles de $\mathbb C$ cartésiens \& polaires
      \item[\checkmark] Diagrammes
      \item[\checkmark] Approximation locale, globale
      \item[\checkmark] Fusion forcée
      \item[\checkmark] Algorithmes de réduction
  \end{itemize}
\end{frame}

\begin{frame}{Modèle théorique}
  \Huge{(Ici, exemple d'un diagramme abstrait additif)}
\end{frame}

\begin{frame}{Modèle théorique}
  \Huge{(Ici, réduction du précédent diagramme)}
\end{frame}

\begin{frame}{Travail réalisé}
  \textbf{Implémentation}
  \begin{itemize}
      \item[\checkmark] Intervalles de $\mathbb C$ cartésiens \& polaires
      \item[\checkmark] Diagrammes : construction, évaluation
      \item[\checkmark] Diagrammes aléatoires
      \item[\checkmark] Fusion forcée
      \item[$\sim$] Algorithmes de réduction
  \end{itemize}
\end{frame}

\begin{frame}{Résultats}
  \begin{columns}
      \begin{column}{0.6\textwidth}
          \begin{tikzpicture}[scale=0.8]
              \begin{axis}[grid=both,
                          xmin=0,ymin=0,
                        xmax=10, ymax=100,
                        axis lines=middle,
                        domain=0:10
                        ]
              \addplot[blue]  {x} node[near end, above]{$y=x$};
              \addplot[black]  {2^x} node[at start, above right]{$y=2^x$};
              \end{axis}
          \end{tikzpicture}
              \end{column}
      \begin{column}{0.4\textwidth}
          L'avantage en nombre de nœuds est \textbf{exponentiel} pour le \textit{proof of concept}
      \end{column}
  \end{columns}
\end{frame}


% 7. Exposez la suite prévue du projet, en particulier en 2e année.
\begin{frame}{Suite du projet}
Suite du projet
\begin{itemize}
    \item \textbf{Implémentation}
    \begin{itemize}
        \item Interface graphique
        \item Benchmarks
    \end{itemize}
    \item \textbf{Ajustements}
    \begin{itemize}
        \item Fonctions d'erreur
        \item Algorithmes de réduction
    \end{itemize}
    \item Nouveaux concepts
    \begin{itemize}
        \item Automates d'arbres
        \item Diagrammes de décisions et applications localement inversibles (LIMDD)
    \end{itemize}
\end{itemize}
\end{frame}

\begin{frame}{Cadre du projet Développements futurs}
  \huge{Cadre du projet \\ Formation future}
\end{frame}

% 3. Présentez le contexte dans lequel vous l'effectuez : éventuellement l'équipe, le laboratoire, les partenaires hors CentraleSupélec, les fonctions des professionnels qui vous entourent.
\begin{frame}{Environnement du projet}
  \begin{columns}
      \begin{column}{0.6\textwidth}
          \begin{itemize}
              \item \textbf{Encadrant} : Renaud Vilmart
              \item \textbf{Équipe} : QuaCS
              \item \textbf{Laboratoire} : Laboratoire Méthodes Formelles
          \end{itemize}
      \end{column}
      \begin{column}{0.4\textwidth}
          \begin{center}
              \includesvg[scale=0.17]{./images/lmf-logo.svg}
              \vspace{0.5cm}
              \includesvg[scale=0.3]{./images/quacs-logo.svg}
          \end{center}
      \end{column}
  \end{columns}
\end{frame}

% 8. Résumez vos choix pour la suite en termes de S8, électifs, international, éventuellement césure
\begin{frame}{Année prochaine}
\begin{center}
  Continuer la formation en \textbf{informatique théorique}
\end{center}
\small{Électifs} % envisagés, choix possibles
\footnotesize{
\begin{itemize}
  \item Génie logiciel orienté objet
  \item Informatique théorique
  \item Calcul haute performance
  \item Modèles et sys. pour la gestion de données
\end{itemize}}
\vspace{1em}
\small{Complément scientifique : métaheuristiques}
\end{frame}

\begin{frame}{Année prochaine}
\textbf{S8} envisagés
\begin{itemize}
  \item Digital Tech Year
  \item S8 à CentraleSupélec
  \begin{itemize}
      \item Continuité du projet
  \end{itemize}
  \item Mobilité internationale
\end{itemize}
\end{frame}

% voire souhaits pour la 3e année
\begin{frame}{Perspectives pour la 3\textsuperscript{e} année}
Dominantes / mentions
\vspace{.5em}
\begin{itemize}
  \item \textbf{Informatique et numérique}
  \begin{itemize}
      \item Sciences du logiciel
      \item Architecture des systèmes informatiques
  \end{itemize}
  \item \textbf{Physique et nanotechnologies}
  \begin{itemize}
      \item Quantum engineering
  \end{itemize}
\end{itemize}
\end{frame}


% 9. Terminez par une conclusion synthétique englobant tous les aspects de votre projet.
\begin{frame}{Conclusion}
\begin{center}
    \huge{Conclusion}
    % Ce projet m'a permis de faire
\end{center}
\end{frame}

\begin{frame}{Conclusion}
    \begin{center}
        \huge{Questions}
    \end{center}
\end{frame}

% Annexes à la présentation
\begin{frame}{Comlément sur les césures}
  Complément sur les césures
  \begin{itemize}
    \item \textbf{Digital Tech Year}
    \begin{itemize}
        \item Semestre au Paris Digital Lab
        \item Semestre en entreprise à l'international
    \end{itemize}
    \item \textbf{Stage}
    \begin{itemize}
        \item Entreprise
        \item Laboratoire
        \item France ou à international
    \end{itemize}
    \item \textbf{Stage en laboratoire}
    \begin{itemize}
        \item En France ou à l'international
    \end{itemize}
  \end{itemize}
\end{frame}

\begin{frame}{Pile technique}
\begin{columns}
    \begin{column}{0.5\textwidth}
        Implémentation
        \begin{itemize}
            \item \textbf{Code} (4,9k lignes)
            \begin{itemize}
                \item Langage C++
                \item LLVM / Clang
                \item Ninja
                \item CMake
            \end{itemize}
            \item \textbf{Tests}
            \begin{itemize}
                \item Google Test
                \item GitHub Actions
            \end{itemize}
        \end{itemize}
    \end{column}
    \begin{column}{0.5\textwidth}
        \begin{center}
            \includesvg[scale=0.23]{./images/cpp-logo.svg}
            \includegraphics[width=.4\textwidth]{./images/llvm-logo.png}

            \includesvg[scale=0.15]{./images/ninja-logo.svg}
            \includegraphics[scale=0.1]{./images/cmake-logo.png}
        \end{center}
    \end{column}
\end{columns}
\end{frame}

\begin{frame}{Pile technique secondaire}
    \begin{columns}
        \begin{column}{0.5\textwidth}
            Mise en forme
            \begin{itemize}
                \item \textbf{Versionnage}
                \begin{itemize}
                    \item Git
                    \item GitHub
                \end{itemize}
                \item \textbf{Documentation} Doxygen
                \end{itemize}
            \end{column}
        \begin{column}{0.5\textwidth}
            \begin{center}
                \includesvg[scale=0.6]{./images/git-logo.svg}
                \includesvg[scale=0.6]{./images/github-logo.svg}

                \includegraphics[scale=0.1]{./images/doxygen-logo.png}
                \end{center}
        \end{column}
    \end{columns}
\end{frame}


\end{document}
