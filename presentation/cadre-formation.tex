\begin{frame}{Cadre du projet Développements futurs}
  \huge{Cadre du projet \\ Formation future}
\end{frame}

% 3. Présentez le contexte dans lequel vous l'effectuez : éventuellement l'équipe, le laboratoire, les partenaires hors CentraleSupélec, les fonctions des professionnels qui vous entourent.
\begin{frame}{Environnement du projet}
  \begin{columns}
      \begin{column}{0.6\textwidth}
          \begin{itemize}
              \item \textbf{Encadrant} : Renaud Vilmart
              \item \textbf{Équipe} : QuaCS
              \item \textbf{Laboratoire} : Laboratoire Méthodes Formelles
          \end{itemize}
      \end{column}
      \begin{column}{0.4\textwidth}
          \begin{center}
              \includesvg[scale=0.17]{./images/lmf-logo.svg}
              \vspace{0.5cm}
              \includesvg[scale=0.3]{./images/quacs-logo.svg}
          \end{center}
      \end{column}
  \end{columns}
\end{frame}

% 8. Résumez vos choix pour la suite en termes de S8, électifs, international, éventuellement césure
\begin{frame}{Année prochaine}
\begin{center}
  Continuer la formation en \textbf{informatique théorique}
\end{center}
\small{Électifs} % envisagés, choix possibles
\footnotesize{
\begin{itemize}
  \item Génie logiciel orienté objet
  \item Informatique théorique
  \item Calcul haute performance
  \item Modèles et sys. pour la gestion de données
\end{itemize}}
\vspace{1em}
\small{Complément scientifique : métaheuristiques}
\end{frame}

\begin{frame}{Année prochaine}
\textbf{S8} envisagés
\begin{itemize}
  \item Digital Tech Year
  \item S8 à CentraleSupélec
  \begin{itemize}
      \item Continuité du projet
  \end{itemize}
  \item Mobilité internationale
\end{itemize}
\end{frame}

% voire souhaits pour la 3e année
\begin{frame}{Perspectives pour la 3\textsuperscript{e} année}
Dominantes / mentions
\vspace{.5em}
\begin{itemize}
  \item \textbf{Informatique et numérique}
  \begin{itemize}
      \item Sciences du logiciel
      \item Architecture des systèmes informatiques
  \end{itemize}
  \item \textbf{Physique et nanotechnologies}
  \begin{itemize}
      \item Quantum engineering
  \end{itemize}
\end{itemize}
\end{frame}
