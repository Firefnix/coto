% 5. Présentez votre travail et vos éventuels résultats.
\begin{frame}{Travail réalisé}
  \textbf{Modèle}
  \begin{itemize}
      \item[\checkmark] Intervalles de $\mathbb C$ cartésiens \& polaires
      \item[\checkmark] Diagrammes
      \item[\checkmark] Approximation locale, globale
      \item[\checkmark] Fusion forcée
      \item[\checkmark] Algorithmes de réduction
  \end{itemize}
\end{frame}

\begin{frame}{Modèle théorique}
  \Huge{(Ici, exemple d'un diagramme abstrait additif)}
\end{frame}

\begin{frame}{Modèle théorique}
  \Huge{(Ici, réduction du précédent diagramme)}
\end{frame}

\begin{frame}{Travail réalisé}
  \textbf{Implémentation}
  \begin{itemize}
      \item[\checkmark] Intervalles de $\mathbb C$ cartésiens \& polaires
      \item[\checkmark] Diagrammes : construction, évaluation
      \item[\checkmark] Diagrammes aléatoires
      \item[\checkmark] Fusion forcée
      \item[$\sim$] Algorithmes de réduction
  \end{itemize}
\end{frame}

\begin{frame}{Résultats}
  Le gain en taille est arbitrairement grand.
\end{frame}
