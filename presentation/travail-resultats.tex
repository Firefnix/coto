% 5. Présentez votre travail et vos éventuels résultats.
\begin{frame}{Travail réalisé}
  \textbf{Modèle}
  \begin{itemize}
      \item[\checkmark] Intervalles de $\mathbb C$ cartésiens \& polaires
      \item[\checkmark] Diagrammes
      \item[\checkmark] Approximation locale, globale
      \item[\checkmark] Erreur
      \item[\checkmark] Fusion forcée
      \item[\checkmark] Algorithmes de réduction
  \end{itemize}
\end{frame}

\begin{frame}{Modèle théorique}

  Exemple : on considère l'état $\begin{pmatrix}
  2+10i \\ 1+4i \\ 2i \\ i
  \end{pmatrix}$

  \vspace{2em}

\end{frame}

\begin{frame}{Modèle théorique} % with colors

  Exemple : on considère l'état $\begin{pmatrix}
  \color{red} 2\color{black}+\color{green}10i \\ \color{red} 1\color{black}+\color{green}4i \\ \color{red}{2i} \\ \color{red}{i}
  \end{pmatrix}$

  \vspace{1em}
  Il existe des \textbf{régularités}.
\end{frame}


\begin{frame}{Modèle théorique}
  Exemple : on obtient le diagramme additif
  \begin{center}
      % https://tikzcd.yichuanshen.de/#N4Igdg9gJgpgziAXAbVABwnAlgFyxMJZARgBoAGAXVJADcBDAGwFcYkQAREAX1PU1z5CKAEyli1Ok1bsAQjz4gM2PASLlxkhizaIQAQQX8VQomRFbpukAB0bAIwgAPGFAAExHpNcBzeEVAAMwAnCABbJA0QHAgkMikddiwjEBDwyJoYpDEQRiwwayh6OAALVxAabRk9ABZkmkZ6exhGAAUBVWEQYKwfEpwUtIjEKKzEeOawKCQAWhqATgb8wuKy6d4g0OGcsYBmGknpxDnF3OX2ItLyhqaW9pM1PR6+gcqrdjsQ+gBjAFYRQZbOKZWKIfZnAoXVbXBLVEAAm7NNodUxPXr9QHpcYgpDgw5IBYbVJAxA7UG7biUbhAA
    \begin{tikzcd}[ampersand replacement=\&, column sep=huge, row sep=large]
      \& D \arrow[rd, "i"] \arrow[ld, "4i"', dashed] \arrow[rd, dashed, bend right=49] \&                                                     \\
    A \arrow[rd, "\frac52"', dashed, bend right=49] \arrow[rd] \&                                                                               \& B \arrow[ld, "2"', dashed] \arrow[ld, bend left=49] \\
      \& \boxed 1                                                                      \&
    \end{tikzcd}
  \end{center}

  \pause
  \textbf{Réduisons} ce diagramme
\end{frame}

\begin{frame}{Modèle théorique}
  On peut forcer la fusion de $A$ et $B$
    % https://tikzcd.yichuanshen.de/#N4Igdg9gJgpgziAXAbVABwnAlgFyxMJZARgBoAGAXVJADcBDAGwFcYkQAREAX1PU1z5CKAEyli1Ok1bsAQjz4gM2PASLlxkhizaIQAQQX8VQomRFbpukAB0bAIwgAPGFAAExHpNcBzeEVAAMwAnCABbJA0QHAgkMikddiwjEBDwyJoYpDEQRiwwayh6OAALVxAabRk9ABZkmkZ6exhGAAUBVWEQYKwfEpwUtIjEKKzEeOawKCQAWhqATgb8wuKy6d4g0OGcsYBmGknpxDnF3OX2ItLyhqaW9pM1PR6+gcqrdjsQ+gBjAFYRQZbOKZWKIfZnAoXVbXBLVEAAm7NNodUxPXr9QHpcYgpDgw5IBYbVJAxA7UG7biUbhAA
  \begin{tikzcd}[ampersand replacement=\&, column sep=huge, row sep=large]
    \& D \arrow[rd, "i"] \arrow[ld, "4i"', dashed] \arrow[rd, dashed, bend right=49] \&                                                     \\
  A \arrow[rd, "\frac52"', dashed, bend right=49] \arrow[rd] \&                                                                               \& B \arrow[ld, "2"', dashed] \arrow[ld, bend left=49] \\
    \& \boxed 1                                                                      \&
  \end{tikzcd}
  $\Rightarrow$
  % https://tikzcd.yichuanshen.de/#N4Igdg9gJgpgziAXAbVABwnAlgFyxMJZABgBpiBdUkANwEMAbAVxiRABEQBfU9TXfIRRkATFVqMWbADrSARhAAeMKAAIAjN14gM2PASJl14+s1aIQAYW7iVAc3hFQAMwBOEALZIR1HBCTq1HIwYFBIACwAnNSmUhaaPC7uXog+IH4B1AxYYOYgUHRwABYqIEEhYYgAtFExknnIPqqybnQAxgCsYmUgDHTBDAAK-PpCIK5YdkU4WkmeSGTp-qnloUg10b05eQXFpXVmbOFYPX0Dw3qCbBNTM4kgbvOIixkrW7lsuyVhq5UAzMR7o8Ui9lmlgmtELUJIcLCcuBQuEA
  \begin{tikzcd}[row sep=large]
    D \arrow[d, "4i"', dashed, bend right=49] \arrow[d, dashed, bend left] \arrow[d, "i", bend left=49] \\
    C \arrow[d, "1", bend left=49] \arrow[d, "{[2, \frac52]}"', dashed, bend right=49]                  \\
    \boxed 1
    \end{tikzcd}
\end{frame}

\begin{frame}{Travail réalisé}
  \textbf{Implémentation}
  \begin{itemize}
      \item[\checkmark] Intervalles de $\mathbb C$ cartésiens \& polaires
      \item[\checkmark] Diagrammes : construction, évaluation
      \item[\checkmark] Diagrammes aléatoires
      \item[\checkmark] Fusion forcée
      \item[$\sim$] Algorithmes de réduction
  \end{itemize}
\end{frame}

\begin{frame}{Résultats}
  On peut toujours plus réduire les diagrammes
  $$\Downarrow$$
  Gain en espace arbitrairement grand (jusqu'à exponentiel)
\end{frame}
