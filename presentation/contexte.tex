% 1. Replacez votre projet dans son contexte : enjeux scientifiques, enjeux sociétaux et/ou économiques (le cas échéant)
\begin{frame}{Contexte}

\begin{center}
  Les besoins en puissance de calcul croissent rapidement
  \includesvg[scale=0.4]{./images/artificial-intelligence-training-computation-by-researcher-affiliation.svg}

  Les algorithmes classiques sont parfois inefficaces

  \pause
  Les \textbf{algorithmes quantiques} permettent de résoudre certains problèmes plus efficacement
  \end{center}
\end{frame}

\begin{frame}{Le besoin}
  Les machines quantiques sont en développement et resteront coûteuses financièrement
  $$\Downarrow$$
  Il y a un besoin d'outils de développement et vérification d'algorithmes quantiques
\end{frame}

\begin{frame}{Simulations}
  Les simulations sont très coûteuses en temps de calcul
  \begin{table}[]
    \begin{tabular}{l|l|l|l}
        \textit{Grover} & Classique & Quantique & Simulation    \\ \hline \rule{0pt}{2.6ex}
    Complexité & $N$       & $\sqrt N$ & {\color{red}$N \sqrt N$}
    \end{tabular}
    % https://arxiv.org/pdf/2005.04635
  \end{table}

  \vspace{1em}

  Elles nécessitent une \textbf{structure de données} adaptée
\end{frame}