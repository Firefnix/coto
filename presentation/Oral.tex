\documentclass[french, 12pt]{beamer}

\usepackage[T1]{fontenc}
\usepackage[french]{babel}
\DecimalMathComma
\usepackage{braket}
\usepackage{svg}
\svgsetup{inkscapelatex=false}

\usepackage{amsmath}
\usepackage{pgfplots}
\pgfplotsset{compat=1.18}
\usepackage{amsfonts}
\usepackage{xfrac}
\newcommand{\somme}{\displaystyle\sum}

\usetheme[compress]{Berlin} % {Boadilla}
\beamertemplatenavigationsymbolsempty
\setbeamertemplate{page number in head/foot}[framenumber]
\setbeamertemplate{frametitle}{}
\useoutertheme{split}

\title{Autour des diagrammes de décision quantiques}
\author{Malo Leroy}
\institute{Parcours recherche -- CentraleSupélec}

\usepackage{tikz}
\usetikzlibrary{positioning}
\usepackage{tikz-cd}

\begin{document}

\begin{frame}
    \titlepage
\end{frame}

% 1. Replacez votre projet dans son contexte : enjeux scientifiques, enjeux sociétaux et/ou économiques (le cas échéant)
\begin{frame}{Contexte}

\begin{center}
Les besoins en puissance de calcul croissent rapidement
\includesvg[scale=0.4]{./images/artificial-intelligence-training-computation-by-researcher-affiliation.svg}

Les algorithmes classiques sont parfois inefficaces

\pause
Les \textbf{algorithmes quantiques} permettent de résoudre certains problèmes plus efficacement
\end{center}
\end{frame}

\begin{frame}
Les machines quantiques sont en développement
$$\Downarrow$$
Il y a un besoin d'outils de développement et vérification d'algorithmes quantiques
\pause

\vspace{1em}
Cela nécessite une \textbf{structure de données} adaptée
\end{frame}

% état de l'art (ce qui a été déjà fait par d'autres)
\begin{frame}{État de l'art}
\textbf{État de l'art}
\begin{itemize}
    \item Interprétation abstraite
    \item Arithmétique des intervalles réels
    \item Diagrammes de décision quantiques
\end{itemize}
\pause
\begin{center}
    Solution : \underline{diagrammes additifs abstraits}
\end{center}
\end{frame}

\begin{frame}{Interprétation abstraite}
    L'\textbf{interprétation abstraite} permet de déterminer des propriétés ou d'accélérer des calculs

    \vspace{1em}
    \underline{Exemple :} signe d'une expression $e = (3 + 2) \times (-5)$
    \begin{align*}
        \text{signe}(e) &= (\text{signe}(3) + \text{signe}(2)) \times \text{signe}(-5) \\
        &= (\oplus + \oplus) \times \ominus \\
        &= \oplus \times \ominus \\
        &= \ominus
    \end{align*}
\end{frame}

\begin{frame}
    \frametitle{Arithmétique des intervalles (1)}

    L'interprétation abstraite est applicable aux \textbf{intervalles réels}

    \begin{alignat*}{2}
    [1, 2] &* [-1, 1] &=& [-2, 2] \\
    [1, 2] &+ [-1, 1] &=& [0, 3] \\
    [1, 2] &\land [-1, 1] &=& [1, 1] \\
    [1, 2] &\lor [-1, 1] &=& [-1, 2]
    \end{alignat*}
\end{frame}

\begin{frame}{Diagrammes de décision}
    Les \textbf{diagrammes de décision} permettent de représenter des fonctions booléennes

\vspace{1em}
% https://tikzcd.yichuanshen.de/#N4Igdg9gJgpgziAXAbVABwnAlgFyxMJZAJgBoAGAXVJADcBDAGwFcYkQAPAfQEYACADoDGEAE58AFN2KDh9MFD7cAzAEoQAX1LpMufIRTLSy6nSat2PTdpAZseAkXIVTDFm0QgpvdVp339J1IeV3MPL2lfGzs9RxRnYlD3dm81a39Yg2QeYySLT3J0210HLJyQmjd8zi4ZIUZ5RRUimNKiMkTKsPZmjVMYKABzeCJQADNRCABbJGcQHAgkADYaRiwwcKh6OAALAaKJ6dmaBaQckAAjGAUkZTmq8O5+AF4+Kz8QQ5nEFfnFxAArKt1pttnsoCAuslPNJnoVVvQrowAAolQKeURYQY7HAHSbfX6nRAAdih1Vh7xsXyQpL+SCBIDWG3YW12+zJjy4yj4r0KH2pJJO-3ODx6XOe70oGiAA
\begin{tikzcd}[ampersand replacement=\&, column sep=small]
    (x_1) \&                                                                \& x_1 \lor (x_2 \land x_3) \arrow[ld, dashed] \arrow[rddd, "x_1 = 1", bend left] \&   \\
    (x_2) \& x_2 \land x_3 \arrow[dd, "x_2=0"', dashed] \arrow[rd, "x_2=1"] \&                                                                                \&   \\
    (x_3) \&                                                                \& x_3 \arrow[ld, "x_3 = 0", dashed] \arrow[rd, "x_3=1"]                          \&   \\
          \& 0                                                              \&                                                                                \& 1
    \end{tikzcd}
\end{frame}

\begin{frame}
    Les \textbf{diagrammes de décision} permettent de représenter des fonctions booléennes

    \vspace{1em}
    \begin{center}
    % https://tikzcd.yichuanshen.de/#N4Igdg9gJgpgziAXAbVABwnAlgFyxMJZARgBoAGAXVJADcBDAGwFcYkQQBfU9TXfQigBMpAMzU6TVu2JceIDNjwEi5MRIYs2iEOTm8lA1aWIap2jtwP8VKMkLNb2XCTCgBzeEVAAzAE4QALZIaiA4EEiiNIxYYBZQ9HAAFm76IP5BITThSGQgAEYwYFCR5FbpAcGIUWERiCIgMXHsCcmp0fSFjAAKfMqCIH5Y7kk4aRlVNTmIACzlE0gz2XUNTfGJKSXzlYvLuZyUnEA
    \begin{tikzcd}[ampersand replacement=\&, column sep=huge, row sep=large]
        \& {} \arrow[ld, dashed] \arrow[rddd, bend left] \&   \\
    {} \arrow[dd, dashed] \arrow[rd] \&                                               \&   \\
        \& {} \arrow[ld, dashed] \arrow[rd]              \&   \\
    0                                \&                                               \& 1
    \end{tikzcd}
    \end{center}
\end{frame}

\begin{frame}{États quantiques}

    Un \textbf{état quantique} est une superposition d'états incompatibles

    $$\ket{\psi} = \alpha \ket{0} + \beta \ket{1} \quad \text{(un qubit)}$$

    \pause
    \begin{center}
        $n$ qubits $\Rightarrow$ $2^n$ états incompatibles
    \end{center}
On note les états sous forme de \textbf{vecteurs}
$$x \ket {01} + y \ket {11} = \begin{pmatrix}
    0 \\ x \\ 0 \\ y
\end{pmatrix}$$
\end{frame}

\begin{frame}{Diagrammes de décision quantiques}
    \begin{columns}
        \begin{column}{0.7\textwidth}
            On peut représenter des états avec des \textbf{diagrammes de décision quantiques}
        \end{column}
        \begin{column}{0.3\textwidth}
            % https://tikzcd.yichuanshen.de/#N4Igdg9gJgpgziAXAbVABwnAlgFyxMJZABgBoAmAXVJADcBDAGwFcYkQAdDgIxgHMsYYGgC29HACcsADwC+ARgAEXLouJcYYKMLGSZskLNLpMufIRRkAzNTpNW7LtwjSYURfMPGQGbHgJE8qQ2NAwsbIggXiZ+5kRkxLZhDpFO-II64lJy5MoqanmqWIUcahpamXpyhrZufPBEoABmEhAiSGQgOBBIQSC8WkgAtFadjIIRIFD0cAAWbtEgLW0dNN1I5DTz9FDskGBsRs2t7YhWaz2InQO7Z53Jk1iLy6fnXZfXmrcj9-aT5CAaIx6LxGAAFUz+CwgKR8WY4QEgcYHdjTOYLWSUWRAA
\begin{tikzcd}[ampersand replacement=\&, column sep=small]
    \begin{pmatrix}2 \\ 0 \\ i \\ 0\end{pmatrix} \arrow[dd, "i", bend left] \arrow[dd, "2"', dashed, bend right] \&    \\
                                                                                                                 \&    \\
    \begin{pmatrix}1 \\ 0\end{pmatrix} \arrow[d, dashed, "1"', bend right]                       \&    \\
    \boxed 1                                                                                                     \& {}
    \end{tikzcd}
        \end{column}
    \end{columns}
\end{frame}

\begin{frame}{État de l'art}
    \textbf{Retour sur l'état de l'art}
    \begin{itemize}
        \item[\checkmark] Interprétation abstraite
        \item[\checkmark] Arithmétique des intervalles réels
        \item[\checkmark] Diagrammes de décision quantiques
    \end{itemize}
    \begin{center}
        On va utiliser ces concepts ensemble,

        avec une nouveauté : l'\textbf{additivité}
\vfill
% https://tikzcd.yichuanshen.de/#N4Igdg9gJgpgziAXAbVABwnAlgFyxMJZARgBoAGAXVJADcBDAGwFcYkQAdDgBWxAF9S6TLnyEU5UsWp0mrdlzQALLAH1iAoSAzY8BImWk0GLNok4dlagEybhusUWtSZJ+ecV9B90fpQBmClc5MwsAGRgAMxwAJywAcyUcehiYiAB3O20RPXFkABYXYxCFSxV1AGpFcttvbIc-ZAA2ItlTUrQvLR1fPIBWIOL2jx4vGRgoePgiUEi0gFskSRAcCCQyEEYsMFCoejglCay5iEXEZdWkZ03t3f3DqGOFpZpLxH86k7OAdle1xAGNx27D2ByOn2eiF+K3+TX4lH4QA
\begin{tikzcd}[ampersand replacement=\&, column sep=small]
    \& \Psi \arrow[ld, dashed] \arrow[d, dashed] \arrow[rd] \&      \& \Leftrightarrow \&               \& \Psi \arrow[ld, dashed] \arrow[rd] \&      \\
\phi_1 \& \phi_2                                               \& \psi \&                 \& \phi_1+\phi_2 \&                                    \& \psi
\end{tikzcd}
\end{center}
\end{frame}

\begin{frame}{Solution proposée}
    \textbf{Retour sur l'état de l'art}
    \begin{itemize}
        \item[\checkmark] Interprétation abstraite
        \item[\checkmark] Arithmétique des intervalles réels
        \item[\checkmark] Diagrammes de décision quantiques
        \item[+] Nouveauté : additivité
    \end{itemize}

    \begin{center}
        Solution : \underline{diagrammes additifs abstraits}
    \end{center}
\end{frame}


% 2. Présentez les objectifs de votre projet.
\begin{frame}{Objectifs}
Objectifs
\begin{itemize}
    \item \textbf{Modèle formel} de diagrammes de décision additifs abstraits
    \item \textbf{Implémentation} du modèle
\end{itemize}

\end{frame}

% 4. Développez la méthodologie que vous avez mise en œuvre durant ces premiers mois, en la justifiant.
\begin{frame}{Méthodologie de travail}
\underline{\textbf{Méthodologie}}
\small{
\begin{center}
% https://tikzcd.yichuanshen.de/#N4Igdg9gJgpgziAXAbVABwnAlgFyxMJZABgBpiBdUkANwEMAbAVxiRBAF9T1Nd9CUARnJVajFmwA6knDAAeOYABEIAYyYBbGGBx08BDp24gM2fQOQBmEdXrNWiENNkLgAUTkwNaBvENceM34iACZSENE7CUdneUUAFQALAEuIACcsGH9jUz4CIgAWcMjxBycZOOAASW8GZK0dPX5-URgoAHN4IlAAMzSIDSQyEBwIJGExeyQwJgYGagY6ACMYBgAFXnM2Xx6cI17+wcQJ0aQwyejyl0VVOjgAAiY4FgY4QwXl1Y2g-McdvYCID6A3G1FOiGsFzKsVcyV0OHusHuDAA5HQ0jh3iBFit1ptgo4Mu1EgDjMCjudwZCotCKq4NNAABe+LE4r7434gIkkkDUFZgKBISzEQHkoVgsaIIpQqR0m7QLK87GfPE-ATYmC7fZAw7ikaS8406azebK3HfPLq-5K-mCiEism6xCUyXSo0xOXAWRwHBvJVs1WW7aagEUDhAA
\begin{tikzcd}[ampersand replacement=\&, column sep=small]
    {} \arrow[r] \& \text{Documentation} \arrow[rr, "\text{cas usuels}"] \arrow[rdd, "\text{état de l'art}"'] \&                                                                 \& \text{Exemples} \arrow[ldd, "\text{modèle}"', bend left] \arrow[rdd, "\text{tests}"] \&                       \\
                 \&                                                                                           \&                                                                 \&                                                                                      \&                       \\
                 \&                                                                                           \& \text{Théorie} \arrow[rr, "\text{code}"] \arrow[ruu, bend left] \&                                                                                      \& \text{Code}
    \end{tikzcd}
\end{center}
}
\end{frame}


% 5. Présentez votre travail et vos éventuels résultats.
\begin{frame}{Modèle théorique}
    \Huge{(Ici, exemple d'un diagramme abstrait additif)}
\end{frame}

\begin{frame}{Modèle théorique}
    \Huge{(Ici, réduction du précédent diagramme)}
\end{frame}

\begin{frame}{Travail réalisé}
\textbf{Modèle}
\begin{itemize}
    \item[\checkmark] Intervalles de $\mathbb C$ cartésiens \& polaires
    \item[\checkmark] Diagrammes
    \item[\checkmark] Approximation locale, globale
    \item[\checkmark] Fusion forcée
    \item[\checkmark] Algorithmes de réduction
\end{itemize}
\end{frame}

\begin{frame}{Travail réalisé}
    \textbf{Implémentation}
    \begin{itemize}
        \item[\checkmark] Intervalles de $\mathbb C$ cartésiens \& polaires
        \item[\checkmark] Diagrammes : construction, évaluation
        \item[\checkmark] Diagrammes aléatoires
        \item[\checkmark] Fusion forcée
        \item[$\sim$] Algorithmes de réduction
    \end{itemize}
\end{frame}

\begin{frame}{Résultats}
    \begin{columns}
        \begin{column}{0.6\textwidth}
            \begin{tikzpicture}[scale=0.8]
                \begin{axis}[grid=both,
                            xmin=0,ymin=0,
                          xmax=10, ymax=100,
                          axis lines=middle,
                          domain=0:10
                          ]
                \addplot[blue]  {x} node[near end, above]{$y=x$};
                \addplot[black]  {2^x} node[at start, above right]{$y=2^x$};
                \end{axis}
            \end{tikzpicture}
                \end{column}
        \begin{column}{0.4\textwidth}
            L'avantage en nombre de nœuds est \textbf{exponentiel} pour le \textit{proof of concept}
        \end{column}
    \end{columns}
\end{frame}

% 7. Exposez la suite prévue du projet, en particulier en 2e année.
\begin{frame}{Suite du projet}
\textbf{Suite}
\begin{itemize}
    \item \textbf{Ajustements}
    \begin{itemize}
        \item Fonctions d'erreur
        \item Algorithmes de réduction
    \end{itemize}
    \item Nouveaux concepts
    \begin{itemize}
        \item Multi-valuation
        \item Carte locale inversible
    \end{itemize}
\end{itemize}
\end{frame}

\begin{frame}{Cadre du projet Développements futurs}
    \huge{Cadre du projet \\ Formation future}
\end{frame}

% 3. Présentez le contexte dans lequel vous l'effectuez : éventuellement l'équipe, le laboratoire, les partenaires hors CentraleSupélec, les fonctions des professionnels qui vous entourent.
\begin{frame}{Environnement du projet}
    \begin{columns}
        \begin{column}{0.6\textwidth}
            \begin{itemize}
                \item \textbf{Encadrant} : Renaud Vilmart
                \item \textbf{Équipe} : QuaCS
                \item \textbf{Laboratoire} : Laboratoire Méthodes Formelles
            \end{itemize}
        \end{column}
        \begin{column}{0.4\textwidth}
            \begin{center}
                \includesvg[scale=0.17]{./images/lmf-logo.svg}
                \vspace{0.5cm}
                \includesvg[scale=0.3]{./images/quacs-logo.svg}
            \end{center}
        \end{column}
    \end{columns}
\end{frame}

% 8. Résumez vos choix pour la suite en termes de S8, électifs, international, éventuellement césure
\begin{frame}{Année prochaine}
\begin{center}
    Continuer la formation en \textbf{informatique théorique}
\end{center}
\small{Électifs} % envisagés, choix possibles
\footnotesize{
\begin{itemize}
    \item Génie logiciel orienté objet
    \item Informatique théorique
    \item Calcul haute performance
    \item Modèles et sys. pour la gestion de données
\end{itemize}}
\vspace{1em}
\small{Complément scientifique : métaheuristiques}
\end{frame}

\begin{frame}{Année prochaine}
\textbf{S8} envisagés
\begin{itemize}
    \item Digital Tech Year
    \item S8 à CentraleSupélec
    \begin{itemize}
        \item Continuité du projet
    \end{itemize}
    \item Mobilité internationale
\end{itemize}
\end{frame}

% voire souhaits pour la 3e année
\begin{frame}{Perspectives pour la 3\textsuperscript{e} année}
Dominantes / mentions
\vspace{.5em}
\begin{itemize}
    \item \textbf{Informatique et numérique}
    \begin{itemize}
        \item Sciences du logiciel
        \item Architecture des systèmes informatiques
    \end{itemize}
    \item \textbf{Physique et nanotechnologies}
    \begin{itemize}
        \item Quantum engineering
    \end{itemize}
\end{itemize}
\end{frame}

% 9. Terminez par une conclusion synthétique englobant tous les aspects de votre projet.
\begin{frame}{Conclusion}
\begin{center}
    \huge{Conclusion}
\end{center}
\end{frame}

\begin{frame}{Conclusion}
    \begin{center}
        \huge{Questions}
    \end{center}
\end{frame}

\begin{frame}{Comlément sur les césures}
Complément sur les césures
\begin{itemize}
    \item \textbf{Digital Tech Year}
    \begin{itemize}
        \item Semestre au Paris Digital Lab
        \item Semestre en entreprise à l'international
    \end{itemize}
    \item \textbf{Stage}
    \begin{itemize}
        \item Entreprise
        \item Laboratoire
        \item France ou à international
    \end{itemize}
    \item \textbf{Stage en laboratoire}
    \begin{itemize}
        \item En France ou à l'international
    \end{itemize}
\end{itemize}
\end{frame}

\begin{frame}{Pile technique}
Technologies utilisées
\begin{itemize}
    \item \textbf{Code}
    \begin{itemize}
        \item Langage C++
        \item CMake
        \item GNU C Compiler
    \end{itemize}
    \item \textbf{Versionnage}
    \begin{itemize}
        \item Git
        \item GitHub
    \end{itemize}
    \item \textbf{Tests}
    \begin{itemize}
        \item Google Test
        \item GitHub Actions
    \end{itemize}
    \item \textbf{Documentation} Doxygen
\end{itemize}
\end{frame}

\end{document}
