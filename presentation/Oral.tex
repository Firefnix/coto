\documentclass[french]{beamer}

\usepackage[T1]{fontenc}
\usepackage[french]{babel}
\DecimalMathComma
\usepackage{braket}

\usepackage{amsmath}
\usepackage{amsfonts}
\usepackage{xfrac}
\newcommand{\somme}{\displaystyle\sum}

\usetheme[compress]{Berlin} % {Boadilla}
\beamertemplatenavigationsymbolsempty
\setbeamertemplate{page number in head/foot}[framenumber]
\setbeamertemplate{frametitle}{}
\useoutertheme{split}

\title{Autour des diagrammes de décision quantiques}
\author{Malo Leroy}
\institute{Parcours recherche -- CentraleSupélec}

\usepackage{tikz}
\usetikzlibrary{positioning}
\usepackage{tikz-cd}

\begin{document}

\begin{frame}
    \titlepage
\end{frame}

% 1. Replacez votre projet dans son contexte : enjeux scientifiques, enjeux sociétaux et/ou économiques (le cas échéant)
\begin{frame}{Contexte}
Contexte :
\begin{itemize}
    \item Besoins grandissants en calcul
    \item Algorithmes inefficaces avec le paradigme classique
    \pause
    \item \textbf{Enjeux scientifiques} : développer des algorithmes quantiques
\end{itemize}
\end{frame}

% état de l'art (ce qui a été déjà fait par d'autres)
\begin{frame}{État de l'art}
État de l'art:
\begin{itemize}
    \item Diagrammes de décision
    \item Interprétation abstraite
    \item Arithmétique des intervalles de $\mathbb{R}^n$
\end{itemize}
\end{frame}

% 2. Présentez les objectifs de votre projet.
\begin{frame}{Objectifs}
Objectifs du projet :
\begin{itemize}
    \item Développer un modèle formel de diagrammes de décision additifs abstraits
    \item Développer une implémentation de ce modèle
\end{itemize}
    
\end{frame}

% 3. Présentez le contexte dans lequel vous l'effectuez : éventuellement l'équipe, le laboratoire, les partenaires hors CentraleSupélec, les fonctions des professionnels qui vous entourent.
\begin{frame}{Environnement du projet}
\begin{itemize}
    \item \textbf{Encadrant} : Renaud Vilmart
    \item \textbf{Équipe} : QuaCS
    \item \textbf{Laboratoire} : Laboratoire Méthodes Formelles (LMF)
\end{itemize}
\end{frame}

% 4. Développez la méthodologie que vous avez mise en œuvre durant ces premiers mois, en la justifiant.
\begin{frame}{Méthodologie de travail}
\textbf{Méthodologie}
\begin{center}
% https://tikzcd.yichuanshen.de/#N4Igdg9gJgpgziAXAbVABwnAlgFyxMJZABgBpiBdUkANwEMAbAVxiRBAF9T1Nd9CUARnJVajFmwA6knDAAeOYABEIAYyYBbGGBx08BDp24gM2fQOQBmEdXrNWiENNkLgAUTkwNaBvENceM34iACZSENE7CUdneUUAFQALAEuIACcsGH9jUz4CIgAWcMjxBycZOOAASW8GZK0dPX5-URgoAHN4IlAAMzSIDSQyEBwIJGExeyQwJgYGagY6ACMYBgAFXnM2Xx6cI17+wcQJ0aQwyejyl0VVOjgAAiY4FgY4QwXl1Y2g-McdvYCID6A3G1FOiGsFzKsVcyV0OHusHuDAA5HQ0jh3iBFit1ptgo4Mu1EgDjMCjudwZCotCKq4NNAABe+LE4r7434gIkkkDUFZgKBISzEQHkoVgsaIIpQqR0m7QLK87GfPE-ATYmC7fZAw7ikaS8406azebK3HfPLq-5K-mCiEism6xCUyXSo0xOXAWRwHBvJVs1WW7aagEUDhAA
\begin{tikzcd}[ampersand replacement=\&, column sep=small]
    {} \arrow[r] \& \text{Documentation} \arrow[rr, "\text{cas usuels}"] \arrow[rdd, "\text{état de l'art}"'] \&                                                                 \& \text{Exemples} \arrow[ldd, "\text{modèle}"', bend left] \arrow[rdd, "\text{tests}"] \&                       \\
                 \&                                                                                           \&                                                                 \&                                                                                      \&                       \\
                 \&                                                                                           \& \text{Théorie} \arrow[rr, "\text{code}"] \arrow[ruu, bend left] \&                                                                                      \& \text{Implémentation}
    \end{tikzcd}
\end{center}
\end{frame}

% 5. Présentez votre travail et vos éventuels résultats.
\begin{frame}{Travail réalisé}
\begin{itemize}
    \item 
\end{itemize}
\end{frame}

% 7. Exposez la suite prévue du projet, en particulier en 2e année.
\begin{frame}{Suite du projet}
\begin{itemize}
    \item 
\end{itemize}
\end{frame}

% 8. Résumez vos choix pour la suite en termes de S8, électifs, international, éventuellement césure
\begin{frame}{Année prochaine}
Choix possibles d'électifs :
\begin{itemize}
    \item Génie logiciel orienté objet
    \item Informatique théorique
    \item Calcul haute performance
    \item Modèles et systèmes pour la gestion des données massives
\end{itemize}
\vspace{1em}
\pause
Choix possibles de S8 :
\begin{itemize}
    \item Digital Tech Year
    \item S8 à CentraleSupélec
    \begin{itemize}
        \item Continuation du projet de parcours recherche
    \end{itemize}
    \item Mobilité internationale
\end{itemize}

\end{frame}

% voire souhaits pour la 3e année
\begin{frame}{Perspectives pour la 3\textsuperscript{e} année}
Dominantes et mentions :
\begin{itemize}
    \item \textbf{Informatique et numérique}
    \begin{itemize}
        \item Sciences du logiciel
        \item Architecture des systèmes informatiques
    \end{itemize}
    \item \textbf{Physique et nanotechnologies}
    \begin{itemize}
        \item Quantum engineering
    \end{itemize}
\end{itemize}
\end{frame}

% 9. Terminez par une conclusion synthétique englobant tous les aspects de votre projet.
\begin{frame}{Conclusion}
\begin{center}
    \huge{Conclusion}
\end{center}
\end{frame}

\begin{frame}{Conclusion}
    \begin{center}
        \huge{Questions}
    \end{center}
\end{frame}

\begin{frame}{Comlément sur les césures}
Complément sur les césures :
\begin{itemize}
    \item \textbf{Digital Tech Year}
    \begin{itemize}
        \item Semestre au Paris Digital Lab
        \item Semestre en entreprise à l'international
    \end{itemize}
    \item \textbf{Stage en entreprise}
    \begin{itemize}
        \item En France ou à l'international
    \end{itemize}
    \item \textbf{Stage en laboratoire}
    \begin{itemize}
        \item En France ou à l'international
    \end{itemize}
\end{itemize}
\end{frame}

\end{document}
