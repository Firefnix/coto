% 5. Présentez votre travail et vos éventuels résultats.
\begin{frame}{Travail réalisé}
  \textbf{Modèle}
  \begin{itemize}
      \item[S6] Intervalles de $\mathbb C$ cartésiens \& polaires
      \item[S6] Diagrammes
      \item[S6] Approximation locale, globale
      \item[S6] Fusion forcée
      \item[S6] Algorithmes de réduction
      \item[S7] Erreur
      \item[S7] Application de portes
  \end{itemize}
\end{frame}

\begin{frame}{Modèle théorique}

  Exemple : on considère l'état $\begin{pmatrix}
    10i+2 \\ 4i+1 \\ 2i \\ i
  \end{pmatrix}$

  \vspace{2em}

\end{frame}

\begin{frame}[noframenumbering]{Modèle théorique} % with colors

  Exemple : on considère l'état $\begin{pmatrix}
  \color{blue}10i \color{black} + \color{cs} 2 \\
  \color{blue}4i  \color{black} + \color{cs} 1 \\
  \color{cs}  2i \\
  \color{cs}  i
  \end{pmatrix}$

  \vspace{1em}
  Il existe des \textbf{régularités}.
\end{frame}

\begin{frame}[noframenumbering]{Modèle théorique} % with colors and schema

  Exemple : on considère l'état $\begin{pmatrix}
    \color{blue} \begin{pmatrix} 10i \\ 4i \end{pmatrix} \color{black} + \color{cs} \begin{pmatrix} 2 \\ 1 \end{pmatrix}
  \\ \color{black} i \color{cs} \begin{pmatrix} 2 \\ 1 \end{pmatrix}
  \end{pmatrix}$

  \vspace{1em}
  Il existe des \textbf{régularités}.
\end{frame}

\begin{frame}[noframenumbering]{Modèle théorique} % with colors and schema and faith

  Exemple : on considère l'état $\begin{pmatrix}
    4i \color{blue} \begin{pmatrix} 5/2 \\ 1 \end{pmatrix} \color{black} + \color{cs} \begin{pmatrix} 2 \\ 1 \end{pmatrix}
  \\ \color{black} i \color{cs} \begin{pmatrix} 2 \\ 1 \end{pmatrix}
  \end{pmatrix}$

  \vspace{1em}
  Il existe des \textbf{régularités}.
\end{frame}

\begin{frame}{Modèle théorique}
  Exemple : on obtient le diagramme additif
  \begin{center}
      % https://tikzcd.yichuanshen.de/#N4Igdg9gJgpgziAXAbVABwnAlgFyxMJZARgBoAGAXVJADcBDAGwFcYkQAREAX1PU1z5CKAEyli1Ok1bsAQjz4gM2PASLlxkhizaIQAQQX8VQomRFbpukAB0bAIwgAPGFAAExHpNcBzeEVAAMwAnCABbJA0QHAgkMikddiwjEBDwyJoYpDEQRiwwayh6OAALVxAabRk9ABZkmkZ6exhGAAUBVWEQYKwfEpwUtIjEKKzEeOawKCQAWhqATgb8wuKy6d4g0OGcsYBmGknpxDnF3OX2ItLyhqaW9pM1PR6+gcqrdjsQ+gBjAFYRQZbOKZWKIfZnAoXVbXBLVEAAm7NNodUxPXr9QHpcYgpDgw5IBYbVJAxA7UG7biUbhAA
    \begin{tikzcd}[ampersand replacement=\&, column sep=huge, row sep=large]
      \& D \arrow[rd, "i"] \arrow[ld, "4i"', dashed] \arrow[rd, dashed, bend right=49] \&                                                     \\
    \color{blue}{A} \arrow[rd, "\frac52"', dashed, bend right=49, blue] \arrow[rd, blue] \&                                                                               \& \color{cs}{B} \arrow[ld, "2"', dashed, cs] \arrow[ld, bend left=49, cs] \\
      \& \boxed 1                                                                      \&
    \end{tikzcd}
  \end{center}

  \pause
  \textbf{Réduisons} ce diagramme
\end{frame}

\begin{frame}{Modèle théorique}
  On peut forcer la fusion de $A$ et $B$
  % https://tikzcd.yichuanshen.de/#N4Igdg9gJgpgziAXAbVABwnAlgFyxMJZARgBoAGAXVJADcBDAGwFcYkQAREAX1PU1z5CKAEyli1Ok1bsAQjz4gM2PASLlxkhizaIQAQQX8VQomRFbpukAB0bAIwgAPGFAAExHpNcBzeEVAAMwAnCABbJA0QHAgkMikddiwjEBDwyJoYpDEQRiwwayh6OAALVxAabRk9ABZkmkZ6exhGAAUBVWEQYKwfEpwUtIjEKKzEeOawKCQAWhqATgb8wuKy6d4g0OGcsYBmGknpxDnF3OX2ItLyhqaW9pM1PR6+gcqrdjsQ+gBjAFYRQZbOKZWKIfZnAoXVbXBLVEAAm7NNodUxPXr9QHpcYgpDgw5IBYbVJAxA7UG7biUbhAA
  \begin{tikzcd}[ampersand replacement=\&, column sep=huge, row sep=large]
    \& D \arrow[rd, "i"] \arrow[ld, "4i"', dashed] \arrow[rd, dashed, bend right=49] \&                                                     \\
  \color{blue}{A} \arrow[rd, "\frac52"', dashed, bend right=49, blue] \arrow[rd, blue] \&                                                                               \& \color{cs}{B} \arrow[ld, "2"', dashed, cs] \arrow[ld, bend left=49, cs] \\
    \& \boxed 1                                                                      \&
  \end{tikzcd}
  $\Rightarrow$
  % https://tikzcd.yichuanshen.de/#N4Igdg9gJgpgziAXAbVABwnAlgFyxMJZABgBpiBdUkANwEMAbAVxiRABEQBfU9TXfIRRkATFVqMWbADrSARhAAeMKAAIAjN14gM2PASJl14+s1aIQAYW7iVAc3hFQAMwBOEALZIR1HBCTq1HIwYFBIACwAnNSmUhaaPC7uXog+IH4B1AxYYOYgUHRwABYqIEEhYYgAtFExknnIPqqybnQAxgCsYmUgDHTBDAAK-PpCIK5YdkU4WkmeSGTp-qnloUg10b05eQXFpXVmbOFYPX0Dw3qCbBNTM4kgbvOIixkrW7lsuyVhq5UAzMR7o8Ui9lmlgmtELUJIcLCcuBQuEA
  \begin{tikzcd}[row sep=large]
    D \arrow[d, "4i"', dashed, bend right=49] \arrow[d, dashed, bend left] \arrow[d, "i", bend left=49] \\
    \color{violet}{C} \arrow[d, violet, bend left=49] \arrow[d, "{[2, \frac52]}"', dashed, bend right=49, violet]                  \\
    \boxed 1
    \end{tikzcd}
\end{frame}

\begin{frame}[noframenumbering]{Modèle théorique}
  On peut forcer la fusion de $A$ et $B$
  % https://tikzcd.yichuanshen.de/#N4Igdg9gJgpgziAXAbVABwnAlgFyxMJZARgBoAGAXVJADcBDAGwFcYkQAREAX1PU1z5CKAEyli1Ok1bsAQjz4gM2PASLlxkhizaIQAQQX8VQomRFbpukAB0bAIwgAPGFAAExHpNcBzeEVAAMwAnCABbJA0QHAgkMikddiwjEBDwyJoYpDEQRiwwayh6OAALVxAabRk9ABZkmkZ6exhGAAUBVWEQYKwfEpwUtIjEKKzEeOawKCQAWhqATgb8wuKy6d4g0OGcsYBmGknpxDnF3OX2ItLyhqaW9pM1PR6+gcqrdjsQ+gBjAFYRQZbOKZWKIfZnAoXVbXBLVEAAm7NNodUxPXr9QHpcYgpDgw5IBYbVJAxA7UG7biUbhAA
  \begin{tikzcd}[ampersand replacement=\&, column sep=huge, row sep=large]
    \& D \arrow[rd, "i"] \arrow[ld, "4i"', dashed] \arrow[rd, dashed, bend right=49] \&                                                     \\
  \color{blue}{A} \arrow[rd, "\frac52"', dashed, bend right=49, blue] \arrow[rd, blue] \&                                                                               \& \color{cs}{B} \arrow[ld, "2"', dashed, cs] \arrow[ld, bend left=49, cs] \\
    \& \boxed 1                                                                      \&
  \end{tikzcd}
  $\Rightarrow$
% https://tikzcd.yichuanshen.de/#N4Igdg9gJgpgziAXAbVABwnAlgFyxMJZABgBpiBdUkANwEMAbAVxiRABEQBfU9TXfIRRkAjFVqMWbAMLdeIDNjwEiZAEzj6zVohAAdPQCMIADxhQABCO7jzAc3hFQAMwBOEALZIyIHBCQi1FpSugAsWADU1tQMdIYwDAAK-MpCIK5YdgAWOCAxWGA6IFB0cFnmeSDxYFBIALShAJw8Lu5eiD5+AUGSRViVsfFJKYJsDDDOudTVtYhNLSBunt2+-ohqPdps0SCDCclKo7rjk5UzSPPyS+2Bq0gbElu6yBsWBm50AMYArBoDcfsRipdBlslNdgUiiUyhVpjAavV5hQuEA
\begin{tikzcd}
  D \arrow[d, "4i+1"', dashed, bend right=49] \arrow[d, "i", bend left=49]           \\
  \color{violet}{C} \arrow[d, bend left=49, violet] \arrow[d, "{[2, \frac52]}"', dashed, bend right=49, violet] \\
  \boxed 1
  \end{tikzcd}
\end{frame}

\begin{frame}{Résultats}
  On peut toujours plus réduire les diagrammes
  $$\Downarrow$$
  Gain en espace arbitrairement grand (jusqu'à exponentiel)
\end{frame}

\begin{frame}{Erreur}
  Comment choisir quels diagrammes fusionner ?
  $$\Downarrow$$
  \textbf{Erreur} : on peut calculer l'erreur induite par un diagramme
\end{frame}

\begin{frame}[noframenumbering]{Portes-bases}
  Les circuits quantiques sont basés sur des \textbf{portes}

  \vspace{1em}
  On les modélise par des \textbf{matrices}

  \pause

  \vspace{1em}
  Exemple : Hadamard
  $$H = \frac{1}{\sqrt{2}}\begin{pmatrix}
    1 & 1 \\
    1 & -1
  \end{pmatrix}$$

  $$H \begin{pmatrix}
    1 \\ 0
  \end{pmatrix} = \frac{1}{\sqrt{2}} \begin{pmatrix}
     1 \\ 1
  \end{pmatrix}$$
\end{frame}

\begin{frame}{Portes-évaluation}
  On veut appliquer une porte $M$ à un diagramme $D$

  \pause

  \vspace{1em}
  Si $\mathcal E(D) = \text{évaluation du diagramme}$

  \vspace{1em}
  On \textit{veut}
  $\mathcal E(M(D)) = M~\mathcal E(D)$

  \pause

  $$D = (\{(l_1, L_1), ..., (l_p, L_p)\}, \{(r_1,R_1), ..., (r_m, R_m)\})$$
\end{frame}

\begin{frame}{Avant Porte}
  \begin{figure}[ht]
    \centering
  % https://tikzcd.yichuanshen.de/#N4Igdg9gJgpgziAXAbVABwnAlgFyxMJZAJgBoAGAXVJADcBDAGwFcYkQAREAX1PU1z5CKcqQCM1Ok1bsAMgH0xPPiAzY8BImPGSGLNohAK0y-uqFEAzDpp6ZhgEqLTqgRuHIALDan72TgFseSRgoAHN4IlAAMwAnCCDEURAcCCRtX3sQRmcaRnoAIxhGAAU3C0NYrDCACxwQPKwwAxAoejga0Jc4hKRk1KQyTJackzzC4rLzTUNGGGj6xub2No6u3hj4xP60xGth9ljc7InS8pmQKtr6jZAe7ZoBxG8Dyvkg8aKz6eFs+ZuVPd0o9dkM7AYwMxGIxPpNzr85gsGiBOvQoOxIMsli0oBAcDh1oCtkh9k8XuCkJDobDvoILojFiiYGiMQQ2NiVniCejuJRuEA
  \begin{tikzcd}[column sep=0.4cm, row sep=2.5cm, ampersand replacement=\&]
    \&     \& D \arrow[lld, "l_1"', dashed] \arrow[ld, "l_p", dashed] \arrow[rd, "r_1"'] \arrow[rrd, "r_m"] \&                                \&     \\
  L_1 \arrow[r, no head, dotted] \& L_p \&                                                                                               \& R_1 \arrow[r, no head, dotted] \& R_m
  \end{tikzcd}
  \end{figure}
  \begin{center}
    {Avant application de la porte}
  \end{center}
\end{frame}

\begin{frame}[noframenumbering]{Pendant portes 1}
  \Huge \color{cs}1.\color{black}

  \begin{figure}[H]
    \tikzcdset{every label/.append style = {font = \large}}
    \centering
    \resizebox{1\textwidth}{!}{
% https://tikzcd.yichuanshen.de/#N4Igdg9gJgpgziAXAbVABwnAlgFyxMJZAFgBoAGAXVJADcBDAGwFcYkQBZACgBEBKEAF9S6TLnyEU5UgEZqdJq3YAZAPoyhIkBmx4CRGbPkMWbRCDVpNo3RKIAmIzRNLzAJXXXtYvZOQBmJwVTdg8AWy8dcX0UAFYglzMQDw1hG2i-ADYExSTwyJ87FAAOHJDzIXkYKABzeCJQADMAJwgIxGkQHAgkQODXMGZGRhpGegAjGEYABUKYkEYYRpwQUawwJKh6OAALaq8WtqRO7qQyfrNB4dGJqdnbecXl1YX1ze29qAPW9pOexGyFyQVxGC1uMzmkgWSxWaRAh1+NFOiEcQMQIJukwhDyhTxWaw27C2u32cIRxyR-0MaIxYKx9wy7DxL0YbyJH1JWnJHUpSHiNKGoLG9MhTJh3yOiGpyNRiWBgsxd1F5mZND29C+5kghIJmwgOBwnKaP16vMQ5zl6IVdKVOLFzzVMA17G1bF1RP1hq+ZJNiH5yMBltpwttjJV4sdzq1BDdr0J5igntJlEEQA
\begin{tikzcd}[column sep=0.4cm, row sep=3cm, ampersand replacement=\&]
  \&     \&                                \&     \& M(D) \arrow[lld, dashed] \arrow[ld, dashed] \arrow[rrd] \arrow[llld, dashed] \arrow[lllld, dashed] \arrow[rd] \&                                \&     \&  \&    \\
L_1 \arrow[r, no head, dotted] \& L_p \& R_1 \arrow[r, no head, dotted] \& R_m \&                                                                                                               \& R_1 \arrow[r, no head, dotted] \& R_m \&  \& {}
\end{tikzcd}
    }
  \end{figure}
\end{frame}

\begin{frame}{Pendant Portes 2}
  \Huge \color{cs}2.\color{black}
  \begin{figure}[H]
    \centering
    \resizebox{1\textwidth}{!}{
      % https://tikzcd.yichuanshen.de/#N4Igdg9gJgpgziAXAbVABwnAlgFyxMJZAFgBoAGAXVJADcBDAGwFcYkQBZACgBEBKEAF9S6TLnyEU5UgEZqdJq3YcA+sHLlBXADIqZA4aOx4CRGbPkMWbRJzUatutAZEgMxiUQBMFmlaW2quoyWgBKei5G4qYoAMy+CtbK9iFc4QC2kW5iJpLIAKwJ-jZ2wDKaOhFCru7ReQBsRYolQeWOKs7VUblEAOxNSYFqMqnh+l3ZHjHIABwDAaUjYSqZQvIwUADm8ESgAGYAThDpSNIgOBBI8YkBYMyMjDSM9ABGMIwACjmetowwezgQE8sGASlB6HAABYbCaHY6nGgXJBkG42O4PJ6vd5fKaSEB-AFA-EgsEQ6FQWFHE6IM5IxCFVFIdGPfFYz7fGL4-6AwwgOHU2mXRCNRmIZmYt7s3HsAk81z8hHnIU+UXi1mSnF1GXcomMEnscFQmG8hU0xFC8yq+4s54ajl42W6-W2Q3kynws1KpD9K0Y9XY+3awkmqmKulzX02tmanq-HUhj2Wukq4pM60SgPSuOEmjQ+gU2yQUFO4suiA4HDG+WhxDXOko1Ni9P+qVa7OA3MwfPsItsYGlkBQcuVikJ6kMukixtq22ZttcnMgPMF8AEPvEgdDitV-Y1n3hvzNNN+2et2MLjtLrsr3slsHD42UQRAA
  \begin{tikzcd}[column sep=0.4cm, row sep=3cm, ampersand replacement=\&]
    \&             \&                                        \&             \& D \arrow[lld, dashed] \arrow[ld, dashed] \arrow[rd] \arrow[rrd] \arrow[llld, dashed] \arrow[lllld, dashed] \arrow[rrrd] \arrow[rrrrd] \&                                        \&             \&                                        \&             \\
  L_1 \arrow[r, no head, dotted] \& L_p \& R_1 \arrow[r, no head, dotted] \& R_m \&                                                                                                                                          \& L_1 \arrow[r, no head, dotted] \& L_p \& R_1 \arrow[r, no head, dotted] \& R_m
  \end{tikzcd}
    }
  \end{figure}
\end{frame}

\begin{frame}[noframenumbering]{Après Portes}
  \Huge \color{cs}3.\color{black}\normalsize
  \begin{figure}[H]
    \centering
    \resizebox{1\textwidth}{!}{
      % https://tikzcd.yichuanshen.de/#N4Igdg9gJgpgziAXAbVABwnAlgFyxMJZAFgBoAGAXVJADcBDAGwFcYkQBZACgBEBKEAF9S6TLnyEU5UgEZqdJq3YcA+sHLlBXADIqZA4aOx4CRGbPkMWbRJzUatutAZEgMxiUQBMFmlaW2quoyWgBKei5G4qYoAMy+CtbK9iFc4QC2kW5iJpLIAKwJ-jZ2wDKaOhFCru7ReQBsRYolQeWOKs7VUblEAOxNSYFqMqnh+l3ZHjHIABwDAaUjYSqZQvIwUADm8ESgAGYAThDpSNIgOBBI8YkBYMyMjDSM9ABGMIwACjmetowwezgQE8sGASlB6HAABYbCaHY6nGgXJBkG42O4PJ6vd5fKaSEB-AFA-EgsEQ6FQWFHE6IM5IxCFVFIdGPfFYz7fGL4-6AwwgOHU2mXRCNRmIZmYt7s3HsAk81z8hHnIU+UXi1mSnF1GXcomMEnscFQmG8hU0xFC8yq+4s54ajl42W6-W2Q3kynws1KpD9K0Y9XY+3awkmqmKulzX02tmanq-HUhj2Wukq4pM60SgPSuOEmjQ+gU2yQUFO4suiA4HDG+WhxDXOko1Ni9P+qVa7OA3MwfPsItsYGlkBQcuVikJ6kMukixtq22ZttcnMgPMF8AEPvEgdDitV-Y1n3hvzNNN+2et2MLjtLrsr3slsHD42UQRAA
  \begin{tikzcd}[column sep=0.4cm, row sep=3cm, ampersand replacement=\&]
    \&             \&                                        \&             \& M(D) \arrow[lld, dashed] \arrow[ld, dashed] \arrow[rd] \arrow[rrd] \arrow[llld, dashed] \arrow[lllld, dashed] \arrow[rrrd] \arrow[rrrrd] \&                                        \&             \&                                        \&             \\
  M_{00}(L_1) \arrow[r, no head, dotted] \& M_{00}(L_p) \& M_{01}(R_1) \arrow[r, no head, dotted] \& M_{01}(R_m) \&                                                                                                                                          \& M_{10}(L_1) \arrow[r, no head, dotted] \& M_{10}(L_p) \& M_{11}(R_1) \arrow[r, no head, dotted] \& M_{11}(R_m)
  \end{tikzcd}
    }
    $$M = \begin{pmatrix}
      M_{00} & M_{01} \\
      M_{10} & M_{11}
    \end{pmatrix}$$
    \end{figure}
\end{frame}

\begin{frame}{Après après Portes}
  \Huge \color{cs}3.\color{black}\normalsize
  \begin{figure}[H]
    \centering
    \resizebox{1\textwidth}{!}{
      % https://tikzcd.yichuanshen.de/#N4Igdg9gJgpgziAXAbVABwnAlgFyxMJZAFgBoAGAXVJADcBDAGwFcYkQBZACgBEBKEAF9S6TLnyEU5UgEZqdJq3YcA+sHLlBXADIqZA4aOx4CRGbPkMWbRJzUatutAZEgMxiUQBMFmlaW2quoyWgBKei5G4qYoAMy+CtbK9iFc4QC2kW5iJpLIAKwJ-jZ2wDKaOhFCru7ReQBsRYolQeWOKs7VUblEAOxNSYFqMqnh+l3ZHjHIABwDAaUjYSqZQvIwUADm8ESgAGYAThDpSNIgOBBI8YkBYMyMjDSM9ABGMIwACjmetowwezgQE8sGASlB6HAABYbCaHY6nGgXJBkG42O4PJ6vd5fKaSEB-AFA-EgsEQ6FQWFHE6IM5IxCFVFIdGPfFYz7fGL4-6AwwgOHU2mXRCNRmIZmYt7s3HsAk81z8hHnIU+UXi1mSnF1GXcomMEnscFQmG8hU0xFC8yq+4s54ajl42W6-W2Q3kynws1KpD9K0Y9XY+3awkmqmKulzX02tmanq-HUhj2Wukq4pM60SgPSuOEmjQ+gU2yQUFO4suiA4HDG+WhxDXOko1Ni9P+qVa7OA3MwfPsItsYGlkBQcuVikJ6kMukixtq22ZttcnMgPMF8AEPvEgdDitV-Y1n3hvzNNN+2et2MLjtLrsr3slsHD42UQRAA
  \begin{tikzcd}[column sep=0.4cm, row sep=3cm, ampersand replacement=\&]
    \&             \&                                        \&             \& M(D) \arrow[lld, dashed] \arrow[ld, dashed] \arrow[rd] \arrow[rrd] \arrow[llld, dashed] \arrow[lllld, dashed] \arrow[rrrd] \arrow[rrrrd] \&                                        \&             \&                                        \&             \\
  M_{00}(L_1) \arrow[r, no head, dotted] \& M_{00}(L_p) \& M_{01}(R_1) \arrow[r, no head, dotted] \& M_{01}(R_m) \&                                                                                                                                          \& M_{10}(L_1) \arrow[r, no head, dotted] \& M_{10}(L_p) \& M_{11}(R_1) \arrow[r, no head, dotted] \& M_{11}(R_m)
  \end{tikzcd}
    }

    \begingroup
    \renewcommand*{\arraystretch}{1.5}
    % your pmatrix expression
        $$\mathcal E(M(D)) = \begin{pmatrix}
  \sum l_i M_{00} \mathcal E (L_i) + \sum r_j M_{01} \mathcal E (R_j) \\
  \sum l_i M_{10} \mathcal E (L_i) + \sum r_j M_{11} \mathcal E (R_j)
\end{pmatrix} = M ~ \mathcal E (D)$$
\endgroup
    \end{figure}
\end{frame}

\begin{frame}{Travail réalisé}
  \textbf{Implémentation}
  \begin{itemize}
      \item[S6] Intervalles de $\mathbb C$ cartésiens \& polaires
      \item[S6] Diagrammes : construction, évaluation
      \item[S6] Fusion forcée
      \item[S6] Algorithmes de réduction
      \item[S7] Diagrammes aléatoires
      \item[S7] Erreur
      \item[S7] Application de portes
      \item[S7] QASM
  \end{itemize}
\end{frame}
