% État de l'art (ce qui a été déjà fait par d'autres)
\begin{frame}{État de l'art}
    \textbf{État de l'art}
    \begin{itemize}
        \item Interprétation abstraite
        \item Arithmétique des intervalles réels
        \item Diagrammes de décision quantiques
    \end{itemize}
    \pause
    \begin{center}
        Solution : \underline{diagrammes additifs abstraits}
    \end{center}
\end{frame}

\begin{frame}{Interprétation abstraite}
    L'\textbf{interprétation abstraite} permet de déterminer des propriétés ou d'accélérer des calculs

    \vspace{1em}
    \underline{Exemple :} signe d'une expression $e = (3 + 2) \times (-5)$
    \begin{align*}
        \text{signe}(e) &= (\text{signe}(3) + \text{signe}(2)) \times \text{signe}(-5) \\
        &= (\oplus + \oplus) \times \ominus \\
        &= \oplus \times \ominus \\
        &= \ominus
    \end{align*}
\end{frame}

\begin{frame}{Intérêts de l'interprétation abstraite}
    L'interprétation abstraite permet de déterminer des propriétés ou d'\textbf{accélérer des calculs}

    \vspace{1em}

    Elle peut être exacte ou \textbf{approximative}
\end{frame}

\begin{frame}
    \frametitle{Arithmétique des intervalles (1)}

    L'interprétation abstraite est applicable aux \textbf{intervalles réels}

    \begin{alignat*}{2}
    [1, 2] &* [-1, 1] &=& [-2, 2] \\
    [1, 2] &+ [-1, 1] &=& [0, 3] \\
    [1, 2] &\land [-1, 1] &=& [1, 1]
    \end{alignat*}

    \small{Le résultat de l'opération est \textbf{le plus petit intervalle} contenant tous les résultats élément par élément}
\end{frame}

\begin{frame}{Fonctions booléennes}

    Une \textbf{fonction booléenne}

    $$f : \{0, 1\}^n \to \{0, 1\}$$

    peut être représentée par une \textbf{table de vérité}

    \begin{center}
        \begin{tabular}{c|c|c|c}
            $x_1$ & $x_2$ & $x_3$ & $f(x_1, x_2)$ \\
            \hline
            0 & 0 & 0 & 0 \\
            0 & 0 & 1 & 0 \\
            0 & 1 & 0 & 0 \\
            0 & 1 & 1 & 1 \\
            1 & 0 & 0 & 1 \\
            1 & 0 & 1 & 1 \\
            1 & 1 & 0 & 1 \\
            1 & 1 & 1 & 1
        \end{tabular}

        \vspace{1em}
        \small{pour $f(x_1, x_2) = x_1 \lor (x_2 \land x_3)$}
    \end{center}
\end{frame}

\begin{frame}{Diagrammes de décision}
    Les \textbf{diagrammes de décision} permettent de représenter des fonctions booléennes

\vspace{1em}
% https://tikzcd.yichuanshen.de/#N4Igdg9gJgpgziAXAbVABwnAlgFyxMJZAJgBoAGAXVJADcBDAGwFcYkQAPAfQEYACADoDGEAE58AFN2KDh9MFD7cAzAEoQAX1LpMufIRTLSy6nSat2PTdpAZseAkXIVTDFm0QgpvdVp339J1IeV3MPL2lfGzs9RxRnYlD3dm81a39Yg2QeYySLT3J0210HLJyQmjd8zi4ZIUZ5RRUimNKiMkTKsPZmjVMYKABzeCJQADNRCABbJGcQHAgkADYaRiwwcKh6OAALAaKJ6dmaBaQckAAjGAUkZTmq8O5+AF4+Kz8QQ5nEFfnFxAArKt1pttnsoCAuslPNJnoVVvQrowAAolQKeURYQY7HAHSbfX6nRAAdih1Vh7xsXyQpL+SCBIDWG3YW12+zJjy4yj4r0KH2pJJO-3ODx6XOe70oGiAA
\begin{tikzcd}[ampersand replacement=\&, column sep=small]
    (x_1) \&                                                                \& x_1 \lor (x_2 \land x_3) \arrow[ld, dashed] \arrow[rddd, "x_1 = 1", bend left] \&   \\
    (x_2) \& x_2 \land x_3 \arrow[dd, "x_2=0"', dashed] \arrow[rd, "x_2=1"] \&                                                                                \&   \\
    (x_3) \&                                                                \& x_3 \arrow[ld, "x_3 = 0", dashed] \arrow[rd, "x_3=1"]                          \&   \\
            \& 0                                                              \&                                                                                \& 1
    \end{tikzcd}
\end{frame}

\begin{frame}
    Les \textbf{diagrammes de décision} permettent de représenter des fonctions booléennes

    \vspace{1em}
    \begin{center}
    % https://tikzcd.yichuanshen.de/#N4Igdg9gJgpgziAXAbVABwnAlgFyxMJZARgBoAGAXVJADcBDAGwFcYkQQBfU9TXfQigBMpAMzU6TVu2JceIDNjwEi5MRIYs2iEOTm8lA1aWIap2jtwP8VKMkLNb2XCTCgBzeEVAAzAE4QALZIaiA4EEiiNIxYYBZQ9HAAFm76IP5BITThSGQgAEYwYFCR5FbpAcGIUWERiCIgMXHsCcmp0fSFjAAKfMqCIH5Y7kk4aRlVNTmIACzlE0gz2XUNTfGJKSXzlYvLuZyUnEA
    \begin{tikzcd}[ampersand replacement=\&, column sep=huge, row sep=large]
        \& {} \arrow[ld, dashed] \arrow[rddd, bend left] \&   \\
    {} \arrow[dd, dashed] \arrow[rd] \&                                               \&   \\
        \& {} \arrow[ld, dashed] \arrow[rd]              \&   \\
    0                                \&                                               \& 1
    \end{tikzcd}

    \vspace{1em}

    On tire parti de la \textbf{structure} de la fonction
    \end{center}
\end{frame}

\begin{frame}{États quantiques}

    Un \textbf{état quantique} est une superposition d'états incompatibles

    $$\ket{\psi} = \alpha \ket{0} + \beta \ket{1} \quad \text{(un qubit)}$$

    \pause
    \begin{center}
        $n$ qubits $\Rightarrow$ $2^n$ états incompatibles
    \end{center}
On note les états sous forme de \textbf{vecteurs}
$$\alpha \ket {01} + \beta \ket {10} = \begin{pmatrix}
    \alpha \\ 0 \\ \beta \\ 0
\end{pmatrix}$$
\end{frame}

\begin{frame}{États quantiques (bis)}
    La représentation usuelle est proche des \textbf{tables de vérité}
    \begin{center}
        \begin{tabular}{c|c|c}
            $x_1$ & $x_2$ & $\braket{x_1 x_2|\psi}$ \\
            \hline
            0 & 0 & $\alpha$ \\
            0 & 1 & 0        \\
            1 & 0 & $\beta$  \\
            1 & 1 & 0
        \end{tabular}

        \vspace{1em}
        \small{pour $\ket \psi = \alpha \ket {00} + \beta \ket {10}$}
    \end{center}
\end{frame}

\begin{frame}{Diagrammes de décision quantiques}
    \begin{columns}
        \begin{column}{0.7\textwidth}
            Les états peuvent être représentés par des \textbf{diagrammes de décision quantiques}

            \vspace{1em}

            On tire parti de la \textbf{structure} de l'état
        \end{column}
        \begin{column}{0.3\textwidth}
            % https://tikzcd.yichuanshen.de/#N4Igdg9gJgpgziAXAbVABwnAlgFyxMJZABgBoAmAXVJADcBDAGwFcYkQAdDgIxgHMsYYGgC29HACcsADwC+ARgAEXLouJcYYKMLGSZskLNLpMufIRRkAzNTpNW7LtwjSYURfMPGQGbHgJE8qQ2NAwsbIggXiZ+5kRkxLZhDpFO-II64lJy5MoqanmqWIUcahpamXpyhrZufPBEoABmEhAiSGQgOBBIQSC8WkgAtFadjIIRIFD0cAAWbtEgLW0dNN1I5DTz9FDskGBsRs2t7YhWaz2InQO7Z53Jk1iLy6fnXZfXmrcj9-aT5CAaIx6LxGAAFUz+CwgKR8WY4QEgcYHdjTOYLWSUWRAA
\begin{tikzcd}[ampersand replacement=\&, column sep=small]
    \begin{pmatrix}2 \\ 0 \\ i \\ 0\end{pmatrix} \arrow[dd, "i", bend left] \arrow[dd, "2"', dashed, bend right] \&    \\
                                                                                                                    \&    \\
    \begin{pmatrix}1 \\ 0\end{pmatrix} \arrow[d, dashed, "1"', bend right]                       \&    \\
    \boxed 1                                                                                                     \& {}
    \end{tikzcd}
        \end{column}
    \end{columns}
\end{frame}

\begin{frame}{Diagrammes de décision quantiques}
    \begin{columns}
        \begin{column}{0.7\textwidth}
            Les états peuvent être représentés par des \textbf{diagrammes de décision quantiques}

            \vspace{1em}

            Dans le pire cela reste \textbf{exponentiel}
        \end{column}
    \begin{column}{0.3\textwidth}
    % https://tikzcd.yichuanshen.de/#N4Igdg9gJgpgziAXAbVABwnAlgFyxMJZABgBoAmAXVJADcBDAGwFcYkQQBfU9TXfQijIBmanSat2AHSkAjCAA8YUAAQBGLjxAZseAkTLExDFm0QdOY5QHN4RUADMAThAC2SMiBwQkamrJgwKCQAWmFPRiwwMxAoejgAC2VNRxd3RHIabw9-QODEcJoTSXMsFJBnNyRMrx9ETwCg0MLxU3ZyEBpGegDGAAU+PUEQJyxrBJxOkEjo9jjE5MtOIA
    \begin{tikzcd}[column sep=huge, row sep=large]
        {} \arrow[dd, "i", bend left] \arrow[dd, "2"', dashed, bend right] \\
\\
        {} \arrow[d, dashed, bend right]\\
        \boxed 1
        \end{tikzcd}
        \end{column}
    \end{columns}
\end{frame}

\begin{frame}{État de l'art}
    \textbf{Retour sur l'état de l'art}
    \begin{itemize}
        \item[\checkmark] Interprétation abstraite
        \item[\checkmark] Arithmétique des intervalles réels
        \item[\checkmark] Diagrammes de décision quantiques
    \end{itemize}
    \begin{center}
        On va utiliser ces concepts \underline{ensemble},

        avec une nouveauté : l'\textbf{additivité}
\vfill
% https://tikzcd.yichuanshen.de/#N4Igdg9gJgpgziAXAbVABwnAlgFyxMJZARgBoAGAXVJADcBDAGwFcYkQAdDgBWxAF9S6TLnyEU5UsWp0mrdlzQALLAH1iAoSAzY8BImWk0GLNok4dlagEybhusUWtSZJ+ecV9B90fpQBmClc5MwsAGRgAMxwAJywAcyUcehiYiAB3O20RPXFkABYXYxCFSxV1AGpFcttvbIc-ZAA2ItlTUrQvLR1fPIBWIOL2jx4vGRgoePgiUEi0gFskSRAcCCQyEEYsMFCoejglCay5iEXEZdWkZ03t3f3DqGOFpZpLxH86k7OAdle1xAGNx27D2ByOn2eiF+K3+TX4lH4QA
\begin{tikzcd}[ampersand replacement=\&, column sep=small]
    \& \Psi \arrow[ld, dashed] \arrow[d, dashed] \arrow[rd] \&      \& \Leftrightarrow \&               \& \Psi \arrow[ld, dashed] \arrow[rd] \&      \\
\phi_1 \& \phi_2                                               \& \psi \&                 \& \phi_1+\phi_2 \&                                    \& \psi
\end{tikzcd}
\end{center}
\end{frame}

\begin{frame}{Solution proposée}
    \textbf{Retour sur l'état de l'art}
    \begin{itemize}
        \item[\checkmark] Interprétation abstraite
        \item[\checkmark] Arithmétique des intervalles réels
        \item[\checkmark] Diagrammes de décision quantiques
        \item[+] Nouveauté : additivité
    \end{itemize}

    \begin{center}
        Solution : \underline{diagrammes additifs abstraits}
    \end{center}
\end{frame}
